\section{Experimental Methods} \label{sec:methods} 

\subsection{Monte Carlo Simulation} \label{sec:MC} 
In order to accurately reconstruct and measure objects that are produced in high-energy physics collisions, a variety of high-level techniques are employed. One of the most widely-used is Monte Carlo simulation, a way of modelling expected real-world observations given certain theoretical predictions. 


Hard scatter


\subsection{Reconstruction and Tagging} \label{sec:Reco} 

Objects 

\subsection{Tracks} \label{sec:Tracks} 

Track reconstruction is essential for identifying and measuring the properties of a wide variety of physics objects, including both muons and hadronic jets.

Initially, the "space points" (spatial coordinates) of potential particle hits are reconstructed using clusters of energy deposits in both the Pixel and SCT detectors. Track candidates are then identified combinatorially using this space-point information (with a minimum of three space-points per track candidate); these track candidates are then passed through an analytic weight-based "ambiguity solver" designed to weed out unphysical track candidates. To aid the ambiguity solver, a neural network is implemented in order to distinguish between isolated and merged clusters of energy deposits. After this, a neural network-based procedure is used to fit the identified tracks. \ref{cite:tracker}.

Fitted tracks from the Pixel detector and SCT can be extended into the TRT if nearby TRT hits are identified. In addition, tracks that are seeded by TRT hits can also be identified and reconstructed using an "outside-in" approach; this helps mitigate the potential loss of real tracks at the ambiguity-solver stage \ref{cite:NEWT}.

After tracks are identified and reconstructed, they are used to identify interaction vertices in the event. Each vertex candidate must contain at least two tracks, each with transverse momentum $p_{T} > 400$ MeV and $|\eta|<2.5$, at least nine hits in the Pixel or SCT detector for tracks with $|\eta|<1.65$ or at least eleven hits for tracks with $|\eta|>1.65$, at least one hit in the first two pixel layers (i.e., the IBL and the inner b-layer), no more than one shared module (i.e. one shared pixel hit or two shared SCT hits), no holes in the Pixel detector, and no more than one hole in the SCT. An iterative combinatorial procedure is then used to fit all compatible tracks to vertex candidates; the primary vertex (PV) representing the initial protom-proton collision is chosen as the vertex with the greatest $\Sigma p_{T}^{2}$ \ref{cite:VertexMeloni}.

However, we note that, for $H \rightarrow \gamma \gamma$ events, a potential lack of charged particles in the final state means that a different primary-vertexing method utilizing ECAL clusters must be considered. This is detailed at length in section \ref{sec:Electrons and Photons}.


\subsection{Clusters} \label{sec:Clusters} 

Photons, electrons and jets are reconstructed using topological clusters of hits ("topo-clusters") in the electromagnetic and hadronic calorimeters. Clusters are reconstructed using a "seed-and-collect" method, where cluster-initiating "seeds" are defined as calorimeter cells with an energy four times greater than the expected noise threshold for that cell  \ref{cite:CERN-EP-2019-145}, \ref{cite:CERN-PH-EP-2015-304}.

After the identification of seed cells, calorimeter cells neighboring the seed cell with recorded energy greater than twice their noise threshold are added to the proto-cluster. If proto-clusters contain more than two local maxima with energy greater than 500 GeV in a single cell, the proto-cluster is split into two in order to account for potential overlap. Initally, seed cells in these split clusters may reside only in the ECAL sampling layers EMB2, EMB3, EME2 and EME3, or the forward calorimeter module FCAL0, after this splitting, clusters are then iteratively split again, with maxima now allowed in the other HCAL and FCAL layers (The first layer of the ECAL is not used to seed topo-clusters in order to reduce the likelihood of producing clusters of noise). In the case of overlap between multiple clusters, cells are assigned to the two cluster candidates with the largest maxima.

\subsubsection{Electrons and Photons} \label{sec:Electrons} 

Electrons and photons are defined using tracks that are matched to topo-clusters in the ECAL. Topo-clusters compatible with EM showers are selected and used to define Regions of Interest; those Regions of Interest are then matched to tracks in the ID. Track candidates are extrapolated into the Regions of Interest using both the measured track momentum and the rescaled momentum measured in the relevant topo-cluster (the latter of which allows for accounting of radiative energy loss in the calorimeter). 

For a track to be considered matched to a topo-cluster, under either momentum-definition-extrapolation, it must fall within $|\Delta \eta| < 0.05$ of its relevant topo-cluster and satisfy $-0.10 < q \times (\phi_{track}-\phi_{cluster}) < 0.05$, where q is the charge of the incident particle. 

Topo-clusters matched to a single charged track are considered to be electron candidates, topo-clusters matched to two oppositely-charged tracks forming a vertex consistent with a photon are considered to be "converted" photon candidates (i.e., clusters resulting from a photon converting into and electron-positron pair in the ID), and topo-clusters matched to no tracks are considered to be "unconverted" photon candidates. In addition, single tracks that have no hits in the innermost layers of the ID are also considered as potential converted photon candidates\ref{cite:CERN-EP-2019-145}. 

Since the start of Run 2, combined topo-clusters called "superclusters" have been implemented in the EM clustering process in order to capture bremsstralung photons and other energy lost in the calorimeter \ref{cite:ATL-PHYS-PUB-2017-022}. For electrons, a supercluster seed must be a cluster with momentum $p_{T} > 1 GeV$ matched to a track with at least four hits, while for photons, a supercluster seed must be a cluster with momentum $p_{T} > 1.5 GeV$. Satellite clusters are then added to the seed to form a supercluster if they fall into a window $\Delta |\eta| \times \Delta\phi = 0.075 \times 0.125$ around the center of the seed cluster. For electrons, an additional satellite cluster search is performed, adding clusters that fall into the window $\Delta |\eta| \times \Delta\phi = 0.075 \times 0.125$ that are matched to the same track as the seed. For converted photons, the $\eta-\phi$ window is not used: all satellite clusters matched to one of the tracks of the converted photon vertex are added to the supercluster.

Following the identification of superclusters and tracks, energy calibration corrections are applied (determined using Monte Carlo simulation for photons and $Z \rightarrow ee$ decays for electrons)\ref{cite:91}, and the photon and electron candidate objects are passed to cluster-shape-based identification algorithms. For both electrons and photons, three identification working points are defined using cut-based multivariate discriminants based on the shower-shape variables.

For photons, the $loose$ ID threshhold is determined based on the following variables:
\begin{itemize}
  \item Acceptance: $|\eta|<2.37$, excluding the calorimeter crack at $1.37 <= |\eta|<1.52$  
  \item $R_{had}$: the ratio of transverse energy deposited in the HCAL to transverse energy deposited in the ECAL for clusters with $0.8 <= |\eta|<1.37$ 
  \item $R_{had1}$: the ratio of transverse energy deposited in the first layer of the HCAL to transverse energy deposited in the ECAL for clusters with $0.8 <= |\eta|<1.37$ 
  \item $R_{eta}$: the ratio of the energy deposited in the ECAL in a $3 \times 7$ rectangle in the $\eta \times \phi$ plane to the energy deposited in the ECAL in a $7 \times 7$ rectangle in the $\eta \times \phi$ plane, both centered on the calorimeter cell with the most deposited energy. 
   \item Lateral shower width $w_{eta2}$: $\sqrt{\frac{\sigma E_{i} \eta_{i}^{2}}{\sigma E_{i}}-{\frac{\sigma E_{i} \eta_{i}}{\sigma E_{i}}}^2}$ (where E is the energy and $\eta$ is the pseudorapidity of cell 'i', summed over all cells in a $3 \times 5$ rectangle centered around the most energetic calorimeter cell. 
\end{itemize}

The $medium$ photon ID threshold is determined based on both passage of the loose threshold and the variable $E_{ratio} = frac{E_{1}-E_{2}}{E_{1}+E_{2}}$, where E_{1} and E_{2} are the leading and subleading energies deposited in calorimeter cells, respectively.

The $tight$ photon ID threshold is determined based on passage of the medium ID threshold and the following shape variables in the strip layer of the ECAL
\begin{itemize}
  \item Lateral Shower Width:$w_{s3} = \sqrt{\frac{\sigma E_{i}(i-i_{max})^{2}}{\sigma E_{i}}}$ (where E is the energy of a strip, '$i_{max}$' is the index of the highest-energy strip, and 'i' is the index of each strip with respect to $i_{max}$ calculated in a $3 \times 2 \eta-\phi$  rectangle centered around the strip with the maximum energy deposit
  \item Total Lateral Shower Width:$w_{s tot} = \sqrt{\frac{\sigma E_{i}(i-i_{max})^{2}}{\sigma E_{i}}}$ (where E is the energy of a strip, '$i_{max}$' is the index of the highest-energy strip, and 'i' is the index of each strip with respect to $i_{max}$ calculated in a $20 \times 2 \eta-\phi$  rectangle centered around the strip with the maximum energy deposit
  \item $\Delta E_{s}$: the difference between the second-largest strip energy and the minimum energy in the strips that lie between the largest- and second-largest strip energies
  \item $f_{1}$: The ratio of the energy in the first layer to the energy in the whole EM cluster.
  \item $f_{side}$: the total energy outside the three central strips but within seven strips, divided by the energy of the three central strips 
\end{itemize}

Each working-point threshold varies in bins of $\eta$. For loose and medium working points, converted and unconverted photons are not treated differently, but for tight working points, they are determined separately. \ref{cite:gammaID CERN-EP-2018-216}

For electrons, the identification process proceeds similarly: working-points are defined using shower-shape variables $f_{1}, E_{ratio}, w_{s tot}, R_{eta}, w_{eta2}, f_{3}, R_{had}, and R_{had1},$ as well as $R_{phi}$ (the ratio of the energy deposited in the ECAL in a $3 \times 3$ rectangle in the $\eta \times \phi$ plane to the energy deposited in the ECAL in a $3 \times 7$ rectangle in the $\eta \times \phi$ plane, both centered on the calorimeter cell with the most deposited energy). Electron ID working-points also include the following track variables:
\begin{itemize}
\item $n_{Blayer}$: Number of hits in the B-layer
\item $n_{Pixel}$: Number of hits in the Pixel
\item $n_{Si}$: Number of hits in the silicon detectors (ID and SCT)
\item $d_{0}$: the transverse impact parameter relative to the beamline
\item $|d_{0} / \sigma(d_{0})|$: the impact parameter significance relative to its uncertainty
\item $\Delta(p)/p = (p-p_{last})/p$: the momentum difference between the track perigee and its endpoint, divided by the momentum at perigee
\item $eProbabilityHT$: the electron probability based on TRT radiation
\item $\Delta \eta_{1}$: the difference in pseudorapidity between the cluster position in the first layer and the matched track
\item $q \times (\phi_{track}-\phi_{cluster})$: where $\phi_{track}$ is the momentum-rescaled track extrapolated from the perigee and $\phi_{cluster}$ is the cluster position in the second ECAL layer.
\item $E/p$: ratio of the cluster energy to the track momentum (used for $E_{T} > 150 GeV$ only) 
\end{itemize}

However, the photon identification is a cut-based process, while the electron identification process relies on a likelihood-based discriminant\ref{cite:elID-CERN-EP-2018-273}.

For the analyses discussed in subsequent chapters, additional photon ID working points are created. These "LoosePrimeN" working points involve photons that pass the 'loose' identification criteria, but fail one or more of N 'tight' identification criteria. The analyses discussed in this dissertation use the LoosePrime4 working point, which is defined as photons passing the Loose identification but failing one or more of the $w_{s3}, f_{side}, \Delta E_{s}$, and $E_{ratio}$ criteria \ref{ANA-EGAM-2018-01}.

To distinguish between photons originating from the hard scatter and photons radiated off of other final-state objects, we consider the relative isolation of identified photons. Photons near large amounts of calorimeter activity are likely to be radiative photons ('non-prompt') radiated from final-state particles after the hard-scatter event, while photons that are isolated are more likely to originate from the hard-scatter.

Two types of isolation variables are considered: calorimetric and track-based.

The calorimetric isolation variable employed in the analyses discussed here is $E_{T}^{cone20} = E_{T,raw}^{isol20} -E_{T}^{core} - C$, where $E_{T,raw}^{isol20}$ is the total calorimeter energy in a cone of $\Delta R = 0.2$ around the electron or photon of interest, $E_{T}^{core}$ is the total calorimeter energy in a $5 \times 7$ rectangle in $\Delta \eta \times \Delta \phi$ around the barycenter of the electron or photon of interest, and C is a correction for pileup and leakage.

The photon track isolation variable employed in the analyses discussed here is $p_{T}^{cone20} = p_{T}^{cone} - p_{T}^{core}$, where $p_{T}^{cone}$ is the total track momentum of all tracks with $p_{T} > 1GeV$ in a cone of $\Delta R = 0.2$ around the photon of interest and $p_{T}^{core}$ is the total track momentum of all tracks with $p_{T} > 1GeV$ matched to the photon candidate. In addition to satifying $p_{T} > 1GeV$, all tracks considered for this metric must also fall within $z= 3mm$ of the primary vertex and have $|\eta|<2.5$.

The electron track isolation variable considered is $p_{T}^{varcone20}$, identical to $p_{T}^{cone20}$ except for the fact that, rather than consider a constant-radius cone of $\Delta R = 0.2$, we consider a cone of radius $\Delta R = max(frac{10}{p_{T}[GeV]}, 0.2)$ \ref{cite:CERN-EP-2019-145}.

In analyses discussed in this dissertation, the diphoton vertex originating from the Higgs decay is often not the hardest vertex. This is because, in processes such as gluon-gluon fusion ($ggF$), many events contain a low final-state track multiplicity \ref{1408.7084}. The diphoton vertex is therefore identified using a Neural Network, trained on variables including the $photon pointing$ (that is, the vertex position in $z$ most compatible with the shower-shapes observed in the ECAL), the $\Delta \phi$ between the vector sum of the track momenta and the diphoton system (as determined by the ECAL), and the scalar momentum sums $p_{T}$ and $p_{T}^2$ for the tracks in each vertex candidate. \ref{HGAMSTXSINT}

The efficiency of the Neural Network diphoton vertex compared to the hardest vertex for a number of targeted physics processes ("STXS bins") for the full Run-2 Couplings analysis is shown in Figure .

\begin{figure}
\end{figure}

\subsubsection{Jets} \label{sec:Jets} 

Hadronic jets are identified from topo-clusters using the $anti-k_{t}$ algorithm.

\subsubsection{b-jets} \label{sec:b-jets} 

\subsubsection{Muons} \label{sec:Muons} 

\subsubsection{MET} \label{sec:MET}

\subsubsection{taus} \label{sec:taus}
We note that, though many physics analyses use $\tau$ leptons, they are not considered in any of the event signatures discussed in this thesis, so we refrain from discussing their reconstruction at length. However, we note that tau leptons can decay both hadronically and leptonically; as a result of this, their reconstruction depends on the reconstruction of electrons, muons, and jets.