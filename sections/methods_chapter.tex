\section{Experimental Methods} \label{sec:methods} 

\subsection{Monte Carlo Simulation} \label{sec:MC} 
In order to accurately reconstruct and measure objects that are produced in high-energy physics collisions, a variety of high-level techniques are employed. One of the most widely-used is Monte Carlo simulation, a way of modelling expected real-world observations given certain theoretical predictions. 



\subsection{Reconstruction and Tagging} \label{sec:Reco} 

Objects 


\subsection{Tracks} \label{sec:Tracks} 

Track reconstruction is essential for identifying and measuring the properties of a wide variety of physics objects, including both muons and hadronic jets.

Initially, the "space points" (spatial coordinates) of potential particle hits are reconstructed using clusters of energy deposits in both the Pixel and SCT detectors. Track candidates are then identified combinatorially using this space-point information (with a minimum of three space-points per track candidate); these track candidates are then passed through an analytic weight-based "ambiguity solver" designed to weed out unphysical track candidates. To aid the ambiguity solver, a neural network is implemented in order to distinguish between isolated and merged clusters of energy deposits. After this, a neural network-based procedure is used to fit the identified tracks. \ref{cite:tracker}.

Fitted tracks from the Pixel detector and SCT can be extended into the TRT if nearby TRT hits are identified. In addition, tracks that are seeded by TRT hits can also be identified and reconstructed using an "outside-in" approach; this helps mitigate the potential loss of real tracks at the ambiguity-solver stage \ref{cite:NEWT}.

After tracks are identified and reconstructed, they are used to identify interaction vertices in the event. Each vertex candidate must contain at least two tracks, each with transverse momentum $p_{T} > 400$ MeV and $|\eta|<2.5$, at least nine hits in the Pixel or SCT detector for tracks with $|\eta|<1.65$ or at least eleven hits for tracks with $|\eta|>1.65$, at least one hit in the first two pixel layers (i.e., the IBL and the inner b-layer), no more than one shared module (i.e. one shared pixel hit or two shared SCT hits), no holes in the Pixel detector, and no more than one hole in the SCT. An iterative combinatorial procedure is then used to fit all compatible tracks to vertex candidates; the primary vertex (PV) representing the initial protom-proton collision is chosen as the vertex with the greatest $\Sigma p_{T}^{2}$ \ref{cite:VertexMeloni}.

However, we note that, for $H \rightarrow \gamma \gamma$ events, a potential lack of charged particles in the final state means that a different primary-vertexing method utilizing ECAL clusters must be considered This is detailed at length in section \ref{sec:Photons}.


\subsection{Photons and Electrons} \label{sec:Photons} 

Photons and electrons are reconstructed using topological clusters of hits ("topo-clusters") in the electromagnetic calorimeter. Clusters are reconstructed using a "seed-and-collect" method, where cluster-initiating "seeds" are defined as cells in the second and third layers of the ECAL with an energy four times greater than the expected noise threshold for that cell (the first layer of the ECAL is not used to seed topo-clusters in order to reduce the likelihood of producing clusters of noise) \ref{cite:CERN-EP-2019-145}.

After the identification of seed cells, ECAL cells neighboring the seed cell with recorded energy greater than twice their noise threshold

\subsubsection{Electrons} \label{sec:Photons} 


\subsubsection{Muons} \label{sec:Muons} 


\subsubsection{Jets} \label{sec:Jets} 

\subsubsection{b-jets} \label{sec:b-jets} 

\subsubsection{Top Reconstruction} \label{sec:top} 

\subsubsection{MET} \label{sec:MET} 

\subsubsection{Taus} \label{sec:taus} 

