The second run of the Large Hadron Collider has allowed for several major milestones to be reached in the field of Higgs physics. Generating $139 fb^{-1}$ of proton-proton collision data from 2015-2018 at a center-of-mass energy of $\sqrt{s} = 13 TeV$, this recent period of collider operation has enabled the ATLAS collaboration to shed new light on the fundamental interactions of the Higgs boson and its couplings. By targeting the diphoton decay channel in particular, physicists have been able to utilize ATLAS's powerful electronic calorimeters to produce clean, precise measurements of Higgs properties.

Two analyses have been explored in depth in this dissertation- the first, a precision measurement of the CP-properties of the top quark Yukawa coupling, and the second, a measurement of Higgs boson production mode cross-sections inclusively, in the $ggF$, $VBF$, $WH$, $ZH$, and $ttH$ channels, and in a number of Simplified Template Cross-Section (STXS) kinematic regions.

In the former analysis, a fully CP-odd top Yukawa coupling was excluded at the level of $3.9 \sigma$. The CP mixing angle was constrained to $|\alpha| > 43^{\circ}$ at 95\% confidence level ($|\alpha| > 63^{\circ} expected$), thus ruling out both the fully odd and maximal-mixing scenarios. In addition, the $ttH$ process attained single-channel observation for the first time (that is, $tt+(H \rightarrow \gamma \gamma)$ was first observed) with an observed significance of $5.2 \sigma$ ($4.4 \sigma$ expected). 

In the latter analysis, the total cross-section times branching ratio ($\sigma \times BR_{\gamma \gamma})$ is measured to be $127\pm 10 fb$, in good agreement with the Standard Model. The $ggF+bbH$ production cross-section is measured to be $104 \pm 11$, the $VBF$ production cross-section is measured to be $10.7^{+2.1}_{-1.9}$, the $WH$ production cross-section is measured to be $6.4^{+1.5}_{-1.4}$, the $ZH$ production cross-section is measured to be $-1.2^{+1.1}_{-1.0}$, and the $ttH+tH$ production cross-section is measured to be $1.2^{+0.4}_{-0.3}$. The compatibility between this measurement and the expected value corresponds to a p-value of 3\%, a 1.9$\sigma$ deviation from the Standard Model. However, when the $WH$ and $ZH$ processes are combined into a single $VH$ process, its cross-section times branching ratio is measured to be $5.9 \pm 1.4$fb, the compatibility between the measurement and the expected value corresponds to a p-value of 50\%, and no significant deviation from the Standard Model is observed. In a near future iteration of the Couplings analysis, a redefined categorization will be introduced to decorrelate the $WH$ and $ZH$ processes, which is expected to improve the agreement with the Standard Model in the five-mode fit.

All production modes in the Couplings analysis are statistically limited. However, the dominant systematic in the analysis is the spurious signal background-mismodelling systematic, which rises to near the value of the statistical uncertainty in the $ggH$ categories. Reducing this mismodelling systematic may thus provide valuable sensitivity improvement on the $ggH$ cross-section measurement. One such tool to do this is Gaussian Process Regression smoothing, which is also expected to be included in the upcoming iteration of the Couplings analysis, and is described at length in the Appendices of this dissertation. Applying Gaussian Process Regression smoothing is projected to improve the overall uncertainty on the $ggH$ measurement by approximately 7\% (in the current categorization scheme), and to improve the uncertainty on the $VBF$ measurement by approximately 2\%. 

This technique has also been successfully implemented in other $H \rightarrow \gamma \gamma$ analyses (notably, a measurement of Higgs differential cross-sections \cite{XSecs}), and is also being investigated in several other analyses currently in preparation (including a search for low-mass diphoton resonances and a search for di-higgs production in the $HH \rightarrow bb \gamma \gamma$ decay channel). In the high-luminosity environment of the upcoming LHC Run 3, the dramatic increase in statistics will lead many analysis channels to become systematically-limited rather than statistically-limited. Thus, developing, validating and implementing uncertainty-reduction techniques such as GPR smoothing will be paramount to the ATLAS physics program.

In the low-statistics regime, the assumptions that GPR relies on (most notably, that every bin follows Gaussian statistics) break down. Thus, one possible useful area of extension of the GPR technique is an extension of the smoothing procedure to the lower-statistical Poisson regimes, perhaps through a process known as a Log Gaussian Cox Process \cite{LogCox}. Additionally, it may be possible to use Gaussian Processes to create the background templates by extrapolating from the data sidebands, rather than merely smoothing templates created using Monte Carlo. However, this would likely need to be tested extensively, as potential sculpting of the background in the signal window may not be properly accounted for using templates generated with this technique.

In addition to the background mismodelling, the final-state heavy-flavor mismodelling uncertainty has been observed to play a nontrivial role in the measurement of the $ggF$, $VBF$ and $VH$ processes. Future analyses targeting $ttH$ and $tH$ will continue to contend with this uncertainty if not addressed, so a dedicated measurement of the $ggF$, $VBF$ and $VH$ processes with heavy-flavor jets is well-motivated. Better understanding the dynamics of these processes may also provide improvement to the modelling of backgrounds in the $H \rightarrow bb$ channel, another useful channel for investigating Higgs interactions.

One notable feature of these two analyses is the competitive limits they place on the single-top associated Higgs production ($tH$) process. This rare process is rapidly approaching discovery potential, and, because it is a statistically limited process, will likely be visible in the early data-taking periods of Run 3 (assuming Standard Model expectation). Observation of the $tH$ process will allow for even more precise measurement of the top Yukawa coupling: in particular, this process is highly sensitive to the sign of $\kappa_{t}$, which is otherwise obscured by a degeneracy as the production cross-section of the $ttH$ process depends only on $|\kappa_{t}|^{2}$. Though the $tH$ process has yet to be observed, it should be noted that the results of the CP analysis do disfavor the $\kappa_{t} < 0$ scenario at greater than the $2 \sigma$ level, one of the most competitive limits placed on the sign of this coupling to date.

Though no BSM physics scenarios are implicated by the results presented in this dissertation, the Higgs sector remains a tantalizing portal to new physics scenarios. Many such models can be interpreted in the context of Effective Field Theories, or EFTs, which place limits on the plausibility of new physics scenarios at higher energies based on observed physics properties at current LHC energies. Interpreting the Higgs production modes in terms of easily-tractable EFT quantities for use by theorists is one of the major motivations for the finely-binned STXS framework implemented in the Couplings measurement.

With Run-2 concluded, the experiments at CERN are currently undergoing upgrades for the 'High-Luminosity-LHC' (HL-LHC) program, which will increase the amount of proton-proton collision data delivered by an order of magnitude over the course of its run. This will, no doubt, open further doors to new Higgs measurements, perhaps including observation of the $tH$ process or other as-yet-unobserved Higgs decays, such as $H \rightarrow HH$. Furthermore, known quantities, such as the parameters discussed in this dissertation, will be able to be measured with unrivalled precision, allowing for increased sensitivity to potential beyond-the-standard-model physics that may be lurking at the TeV scale. The diphoton channel will no doubt continue to prove a useful measurement channel in the future, and subsequent LHC measurements will in all likelihood build upon much of the progress that was made during this second LHC run.
