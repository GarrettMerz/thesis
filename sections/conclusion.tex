The second run of the Large Hadron Collider has allowed for several major milestones to be reached in the field of Higgs physics. Generating $139 fb^{-1}$ of proton-proton collision data from 2015-2018 at a center-of-mass energy of $\sqrt{s} = 13 TeV$, this recent period of collider operation has enabled the ATLAS collaboration to shed new light on the fundamental interactions of the Higgs boson and its couplings. By targeting the diphoton decay channel in particular, physicists have been able to utilize ATLAS's powerful electronic calorimeters to produce clean, precise measurements of Higgs properties.

Two analyses have been explored in depth in this dissertation- the first, a precision measurement of the CP-properties of the top quark Yukawa coupling, and the second, a measurement of Higgs boson production mode cross-sections inclusively, in the $ggF$, $VBF$, $WH$, $ZH$, and $ttH$ channels, and in a number of Simplified Template Cross-Section (STXS) kinematic regions.

In the former analysis, a number of multivariate algorithms were employed to aid in categorization, including Boosted Decision Trees for top quark reconstruction, background rejection, and CP-even vs. CP-odd discrimination. A fully CP-odd top Yukawa coupling was excluded at the level of $3.9 \sigma$. The CP mixing angle was constrained to $|\alpha| > 43^{\deg}$ at 95\% confidence level ($|\alpha| > 63^{\deg} expected$), thus ruling out both the fully odd and maximal-mixing scenarios. In addition, the $ttH$ process attained single-channel observation for the first time (that is, $tt+(H \rightarrow \gamma \gamma)$ was first observed) with an observed significance of $5.2 \sigma$ ($4.4 \sigma$ expected). 

In the latter analysis, a multiclass BDT is designed to target STXS truth bins. 88 regions are defined, and are combined to in a number of ways to produce different observables. The total cross-section times branching ratio ($\sigma \times BR_{\gamma \gamma})$ is measured to be $127\pm 10 fb$, in good agreement with the Standard Model. The observed (expected) significance values for the $VBF$, $WH$, and $ttH+tH$ processes are 7.5 (6.1) $\sigma$, 5.6 (2.8) $\sigma$, and 4.7 (5.0) $\sigma$, respectively. In addition, an upper limit of 8.2 times the SM prediction is set for the $tH$ process, the most stringent limit to date. When category combination is taken into account, no substantial deviation from the Standard Model was observed. In a near future iteration of this analysis, updates will include a redefined categorization to decorrelate the $WH$ and $ZH$ processes, a measurement of couplings using the Kappa-framework, and the introduction of a novel Gaussian Process Regression smoothing technique to reduce spurious signal uncertainty.

With Run-2 concluded, the experiments at CERN are currently undergoing upgrades for the 'High-Luminosity-LHC' (HL-LHC) program, which will increase the amount of proton-proton collision data delivered by an order of magnitude over the course of its run. This will, no doubt, open further doors to new Higgs measurements, perhaps including observation of the $tH$ process or rare as-yet-unobserved Higgs decays. Furthermore, known quantities, such as the parameters discussed in this dissertation, will be able to be measured with unrivalled precision, allowing for increased sensitivity to potential beyond-the-standard-model physics that may be lurking at the TeV scale. The diphoton channel will no doubt prove a useful measurement channel in the future as well, and subsequent LHC measurements will in all likelihood build upon much of the progress that was made during this second LHC run.