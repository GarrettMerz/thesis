\section{Data, Monte Carlo, and HGam Pre-selection} \label{sec:DataMC}

\subsection{Data} \label{sec:Data}

In both Run-2 HGam analyses discussed in this dissertation, we use the full LHC Run-2 dataset, consisting of $139.0 \plusminus 2.4 fb^{-1}$ of proton-proton collisions with a center of mass energy $\sqrt{s} = 13 TeV$ recorded by the ATLAS detector between 2015 and 2018 \ref{Jennet106}. Figure XXX shows the instantaneous luminosity gathered by the detector as a function of time. 

The trigger used to select events is the diphoton trigger HLT_g35_loose_g25_loose (for 2015-2016 data) and HLT_g35_medium_g25_medium (for 2017-2018 data). Both triggers require two photon candidates, one with a transverse energy $E_{T}$ of at least 35 GeV and the other with transverse energy of at least 25 GeV. The 2015-2016 trigger requires two photons that pass the "loose" ID requirement, while the 2017-2018 trigger requires two photons that pass the "medium" ID requirement (the cut was tightened due to increased luminosity and pileup). The trigger was calibrated using radiative Z decays, and is observed to be greater than 95\% efficient for each photon as long as it is 5 GeV above the trigger threshold \ref:cite{TriggerPerformance}.

Figure: EffCurve

\subsection{Higgs Preselection and Data CRs} \label{sec:Preselection} 

In both analyses discussed in this dissertation, we use the following preselection to define the $H \rightarrow \gamma \gamma$ signal region. These requirements are applied for both data and Monte Carlo simulation. 

\begin{itemize}
\item $\gamma\gamma Preselection$: The event is required to contain two photons passing the loose isolation and ID requirements.
\item $PV$: At least one primary vertex is required to be identified in the event.
\item $Trigger-Matching$: The leading two photons observed are required to match those identified by the trigger.
\item $Relative p_{T}$: The leading and subleading photons are required to have $p_{T}/m\gamma\gamma$ larger than 0.35 and 0.25, respectively.
\item $TI$: For events in the signal-region, we require both photons to pass the "Tight" Photon ID and Isolation requirements (for events in the 'NTI' data control region, this selection criterion is ignored). 
\item $Invariant Mass$: The invariant mass of the diphoton system must satisfy $105 GeV < m_{\gamma\gamma} < 160 GeV$
\end{itemize}

In order to accurately model the continuum diphoton background, it is often useful to invert the $TI$ requirement- that is, we construct a data sample consisting of those events passing all other preselections, but for which one or more photons does not pass either or both of the "tight" isolation and ID requirements. Data control samples in this not-tight-isolated "$NTI$" region allow us to model the kinematic properties of objects in the event other than the diphoton system, such as top quark jet variables, with some amount of accuracy.

Additionally, it is useful to consider the $TI$ or $NTI$ data sidebands- this consists of the set of events passing either $TI$ or $NTI$ selection with either $105 GeV < m_{\gamma\gamma} < 120 GeV$ or $130 GeV < m_{\gamma\gamma} < 160 GeV$. This allows us to model the shape of the background while safely away from the Higgs signal peak near 125 GeV. This is detailed further in the subsequent analysis-specific chapters.

\subsection{Nominal Monte Carlo Samples} \label{sec:NominalMC} 

In order to generate Monte Carlo events, several software programs must work in concert. In particular, a generator 







All Monte Carlo samples are generated using the Geant4 full simulation with the exception of the QCD continuum diphoton samples used for background modelling, which uses the Geant4 fast simulation due to the large numbers of events. For all MC samples, pile-up interactions were simulated by overlaying each Monte Carlo event with a different number of minimum-bias events simulated using Pythia 8.186 with the ATLAS "A3" tune [8]



















\subsection{ttHCP Monte Carlo Samples} \label{sec:ttHCPMC} 

In the $ttH CP$ analysis, an Effective Field Theory (EFT) setting a cutoff scale of 1 TeV, below which no new BSM particles coupling to the Higgs exist, is used to generate Monte Carlo samples. The EFT used is the Higgs Characterization (HC) model \ref{cite: HC}, implemented in the  MadGraph5_aMC@NLO generator \ref{cite: Madgraph}. As previously mentioned, the top-Higgs interaction term of the Lagrangian in the presence of CP-violation can be parameterized as

\begin{equation}
\mathcal{L} = \kappa_{t} g_{t} \bar{t} (cos(\alpha)+ sin(\alpha) i \gamma^{5} )th
\end{equation}

where $g_{t} = \frac{-m_{t}}{v} = \frac{-173.26 GeV}{246 GeV} = -0.703$ , $\kappa_{t}$ is the dimensionless coupling-strength term ($\kappa_{t}= 1$ in the Standard Model), and $\alpha$ is an angle that parameterizes the CP-mixing strength ($\alpha = 0$ in the Standard Model, $\alpha = \frac{\pi}{2}$ in the fully CP-odd case). The treatment of the $H \rightarrow \gamma \gamma$ and $ggH$ dependence on $\alpha$ is handled in several different ways, as discussed in \ref{sec:ttHCPresults}.

For all Monte Carlo samples, the renormalization ($μ_{R}$) and factorization ($μ_{F}$) scales are defined as the scalar sum of the transverse masses of all final-state particles divided by two (i.e. $H_{T}/2$). $t\bar{t}H$ and $tWH$ samples are generated using the five-flavor scheme, while the four-flavor scheme is used for the $tHjb$ process. The Standard Model cross-sections for all process are normalized to those given in the CERN Higgs Yellow Report 4 \ref{cite:yellowreport}, in which fixed scales and the five-flavor scheme are used. Those cross-sections are calculated at NLO QCD accuracy (without electroweak correction) for the $tHjb$ and $tWH$ processes, while $t\bar{t}H$ is calculated at both NLO QCD and NLO Electroweak accuracies. K-factors are then computed to scale the Higgs Characterization Monte Carlo cross-sections to the Yellow Report cross-sections. The obtained K-factors are shown to be similar for different CP mixing angles; thus, the K-factors derived for the SM case can be safely used for the various samples with different $\alpha$ values.



\subsection{SMEFT Monte Carlo Samples} \label{sec:SMEFTMC} 







