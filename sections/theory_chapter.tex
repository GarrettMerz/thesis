\usepackage{ amssymb }

\section{The Standard Model} \label{sec:SM} 

The Standard Model of Particle Physics is arguably one of the crowning achievements of the last century of physics research. Though it is doubtless incomplete (it does not, for instance, explain the dark energy or dark matter observed in cosmological experiments, nor does it provide a satisfactory quantum-mechanical model of gravity), all predictions it has made have yet to be falsified \ref{Peskin}, \ref{Gordy}, \ref{FTGA}, \ref{Griffiths}.

The Standard Model is a quantum field theory, meaning that it describes the behavior of fields (physical quantities that are defined at all points in spacetime; common examples of fields include electric and magnetic fields) and their discrete, quantized excitations, referred to as particles (common examples of particles are the electron and the photon, which are excitations of the "electron field" and the electromagnetic field respectively). More about the mathematics of field dynamics will be discussed in section \ref{sec:Lagrangians}.

The Standard Model divides these fundamental particles (and their corresponding fields) into two major categories: the bosons and the fermions. Fermions carry half-integer spin, while bosons carry integer spin. In a general sense, the elementary fermions can be thought of as the particles that comprise matter, while the elementary bosons can be thought of as the particles that correspond to the behavior of the four fundamental forces (electromagnetism, the strong nuclear force, the weak nuclear force, and gravity).

The elementary fermions of the Standard Model all have spin $1/2$. They are divided into two major subgroupings, quarks and leptons; each of these is further divided into three generations. Each generation of quarks contains one up-type quark (up, charm, or top) and one down-type quark (down, strange, and bottom), while each generation of leptons contains one charged lepton (electron, muon, or tau) and one neutrino (electron neutrino, muon neutrino, or tau neutrino). Fermions of successive generations behave similarly to each other, though those of each subsequent generation are more massive than the last.

The quarks are the only fermions that undergo both the strong and weak nuclear interactions, and are also the only particles in the Standard Model that have fractional electromagnetic charge ($+2/3$ for up-type quarks and $-1/3$ for down-type quarks). However, because quarks are never found in isolation and are always bound to other quarks in composite states called hadrons, all observable quark final states have integer charge. Conversely, the charged leptons all have an electromagnetic charge of $-1$, while neutrinos are chargeless. While the quarks and charged leptons have precisely-measured masses, the mass of the neutrinos is vanishingly small, and, to date, only the differences between the different neutrino masses have been conclusively measured.

In addition to these 12 fermion species, each fermion species has a corresponding antimatter "antifermion" species, which has the opposite charge and parity quantum numbers. This is discussed at length in section \ref{sec:CPT}.

The bosons of the standard model are the "messengers" through which the fundamental forces operate (with the exception of the Higgs boson, which is detailed at length in section \ref{sec:EWSB}). The photon and the gluons are massless, while the two W-bosons, the Z-boson, and the Higgs boson are all massive. The photon is the mediator of the electromagnetic force and couples to charged particles; the gluon is the mediator of the strong nuclear force and couples to particles with "color charge", and the W and Z bosons are the mediators of the weak nuclear interaction, coupling to all fermions with left-handed parity and antifermions with right-handed parity. In addition, the three weak bosons ($W^{+}$, $W^{-}$, and $Z$) can all couple to each other, as can the gluons.

The strong nuclear interaction binds quarks together, while the weak interaction governs the decay of one species of fermion into another. Weak interactions operate primarily on fermion doublets, coupling each up-type quark to its corresponding down-type quark and each charged lepton to its corresponding neutrino. However, some intergenerational coupling does occur, the rarity of which is governed by a matrix of coefficients called the Cabibbo-Kobayashi-Makasawa (or CKM) matrix. 

The Higgs boson is the only scalar (spin-0) boson in the Standard Model. Though it does not mediate any force directly, its existence is a consequence of the unification of the electromagnetic and weak nuclear interactions into one "electroweak" interaction at high energy scales. Without the role of the Higgs boson in this process, the fermions, the W-bosons, and the Z-boson would all be massless; thus, the Higgs can be said to "give mass" to the particles of the Standard Model. It couples to all massive particles in the Standard Model, namely, the fermions and the W and Z bosons.

\subsection{sec:Lagrangians, Fields, and Gaug}

In order to fully explain the Higgs mechanism, we must first discuss the mathematical language of quantum field theory. Both quantum and classical field theories are often discussed using Lagrangian dynamics, where the Lagrangian is defined as  $\mathcal{L} = T - V$, the difference of the kinetic and the potential energy. Physical properties will always evolve in such a way that the integral of this property with respect to time, $\mathcal{S} = \int \mathcal{L} dt$, called the action, is a constant. Lagrangians are also incredibly useful in that they give rise to conservation laws: Noether's Theorem states that operations performed on a system that do not change the Lagrangian are each associated to conserved quantities of a system (i.e., systems with translationally invariant Lagrangians must respect conservation of momentum, systems with temporally-invariant Lagrangians must respect conservation of energy, etc.).

A variety of types of fields exist, but we will discuss three at length: Klein-Gordon fields, which are scalar fields (single quantities defined everywhere), Dirac spinor fields (vector quantities defined everywhere), and gauge fields (additional vector fields that must be introduced in order to preserve certain physical symmetries. We begin with the Klein-Gordon field.

Klein-Gordon fields, one of the simplest examples of a field, are real scalar-valued quantities often denoted using the symbol $\phi$. A non-interacting "free" Klein-Gordon field evolves according to the Lagrangian:

\begin{equation}
\mathcal{L} = T - V 
=\frac{1}{2} \frac{\partial^{2} \phi}{\partial t^{2}} - \frac{1}{2} (\nabla \phi)^{2} - \frac{1}{2} m^{2} \phi^{2}
=-\frac{1}{2}(\partial^{\mu}\phi)(\partial_{\mu}\phi)- \frac{1}{2} m^{2} \phi^{2}
\end{equation}

where we utilize Einstein sum notation in the last line to compress the four derivatives in the preceding expression into a single shorthand symbol. The Lagrangian of an interacting Klein-Gordon field would look similar, but would possess additional $V$ potential terms depending on the nature of the interaction. As the only scalar in the Standard Model, the observable Higgs boson is the only elementary particle to follow an interacting Klein-Gordon field equation.

While the concept of a complex-valued scalar field does not directly correspond to any of the physical elementary particles that are present in the Standard Model, it plays an important role in the understanding of the Higgs mechanism. A complex-valued scalar field behaves similarly to a real-valued one, with the Lagrangian

\begin{equation}
\mathcal{L} = (\partial^{\mu}\phi^{*})(\partial_{\mu}\phi) - \frac{1}{2} m^{2} \phi^{*} \phi
\end{equation}

Where * denotes the complex conjugate of the scalar field.

Finally, Dirac fields describe all Standard Model fermions (with the possible exception of neutrinos). They are vector fields as opposed to scalar fields, and behave according to the Lagrangian:

\begin{equation}
\mathcal{L} = \bar{\psi}(i \gamma^{mu} \partial_{mu} - m) \psi
\end{equation}

where $\gamma^{mu}$ denotes the set of Dirac gamma matrices, and $\bar{\psi} = \psi^{\dagger} \gamma^0$ denotes the transpose of the complex conjugate of the vector field multiplied by one of these matrices, defined as such in order to preserve invariance under relativistic Lorentz boosts. 

A four-component Dirac vector field can be written in a variety of representations, but one of the most useful is that of a doublet of two two-component Weyl spinors, one left-handed and one right-handed, that is, $\psi = \begin{pmatrix} \psi_L \\ \psi_R \end{pmatrix}$. Each of these components transforms differently under Lorentz boosts.

In order to discuss the gauge fields corresponding to the Standard Model bosons, we must first discuss the concept of gauge symmetries. Consider a single Dirac vector field described by equation \ref{eq:Dirac}. We transform the field by rotating it by a constant phase $\alpha$:  $\psi \rightarrow e^{i \alpha} \psi $, $\bar{\psi} \rightarrow e^{-i \alpha} \bar{\psi} $.

\begin{equation}
\mathcal{L} = - (e^{-i \alpha} \bar{\psi})(i \gamma^{mu} \partial_{mu} - m) (e^{i \alpha} \psi)
= - (e^{-i \alpha} e^{i \alpha})\bar{\psi} (i \gamma^{mu} \partial_{mu} - m)\psi
= -\bar{\psi}(i \gamma^{mu} \partial_{mu} - m) \psi
\end{equation}

We see that we recover the original Dirac Lagrangian. Thus, the lagrangian is invariant under a constant phase rotation. A transformation of this charater is called a global gauge transformation. A one-dimensional rotation is a unitary transformation, so we call this a global $U(1)$ gauge symmetry.

We next consider the concept of a local gauge transformation, that is, one in which the phase $\alpha$ may vary with position. The field transforms, as before, like  $\psi \rightarrow e^{i \alpha (x)} \psi $, $\bar{\psi} \rightarrow e^{-i \alpha (x)} \bar{\psi} $. However, we note that the dependence on position means that we can no longer factor out the exponentials, and we thus have

\begin{equation}
\mathcal{L} = - (e^{-i \alpha (x)} \bar{\psi})(i \gamma^{mu} \partial_{mu} - m) (e^{i \alpha (x)} \psi)
=-i e^{-i \alpha (x)} \bar{\psi} \gamma^{mu} (e^{i \alpha (x)} \partial_{mu} \psi + i e^{i \alpha (x)} \psi \partial_{mu} \alpha -m \bar{psi}\psi
=-\bar{\psi}(i \gamma^{mu} \partial_{mu} + \gamma^{mu} \partial_{mu} \alpha - m) \psi
\end{equation}

i.e., the Lagrangian is not invariant under this sort of transformation. However, local gauge invariance is an important physical symmetry, so in order to attempt to preserve it, we add an additional term to the original Lagrangian: an extra vector field $A_\mu$ that transforms like $A_\mu \rightarrow A_\mu - \frac{1}{q} \partial_{mu} \alpha(x) $ for a constant q. We also define a new operator, called the covariant derivative: $D_{mu} = \partial_{mu}+ iqA_{mu} $. Given how $A_{mu}$ transforms, we see that $D_{mu}$ transforms like

\begin{equation}
D_{mu} = \partial_mu + iqA_{mu}
\rightarrow \partial_mu + iq(A_{mu}-frac{1}{q}\partial_{mu}\alpha(x))
= \partial_mu + iqA_{mu}-i\partial_{mu}\alpha(x)
=D_{mu}-i\partial_{mu}\alpha(x)
\end{equation}

Let us replace the $\partial_{mu}\phi$ terms in the initial Lagrangian with $\D_{mu}\phi$ and transform.

\begin{equation}
\mathcal{L} = - (e^{-i \alpha (x)} \bar{\psi})(i \gamma^{mu} (D_{mu}-i\partial_{mu}\alpha(x)) - m) (e^{i \alpha (x)} \psi)
=-(e^{-i \alpha (x)} \bar{\psi})(i \gamma^{mu} (\partial_mu + iqA_{mu}-i\partial_{mu}\alpha(x))-m)(e^{i \alpha (x)} \psi)
=-\bar{\psi}(i \gamma^{mu} \D_{mu} + \gamma^{mu} \partial_{mu} \alpha - \gamma^{mu} \partial_{mu} \alpha - m) \psi
=-\bar{\psi}(i \gamma^{mu} \D_{mu} - m) \psi
\end{equation}

Thus, by introducing an additional vector field that corresponds to the local U(1) gauge symmetry, we have restored the invariance of our lagrangian. Physically, this field is analogous to the introduction of electromagnetism to our single Dirac fermion model, with the $A_mu$ field playing the role of the photon: it is a vector quantity and so has spin-1, it must be massless (as adding an $A_mu$ mass term to the Lagrangian would violate the symmetry again), and couples to the fermion fields according to their electromagnetic charge. We note that we can also still preserve invariance if we add an additional term $L_{kin} = -\frac{1}{4}F^{\mu \nu} F_{\mu \nu}$ to the Lagrangian, where $F_{\mu \nu} = \partial_{\mu} A_{\nu} - \partial_{\nu} A_{\mu}$: this corresponds to the energy inherent in electromagnetic fields themselves. 

Each of the fundamental forces in the Standard Model can be understood in terms of these sorts of gauge symmetries. The photon is the particle excitiation of the electromagnetic field, which corresponds to a one-dimensional rotation "U(1)" transformation. In order to be invariant under three dimensional transformations of the type "SU(3)" (the set of all volume-preserving, $\bar{\psi}\psi$-preserving transformations in a 3D vector space), we must add eight new vector fields (these are the eight gluons), leading to the incorporation of the strong interaction into the Standard Model.

Similarly, in order to be invariant under two-dimensional transformations of the type "SU(2)" (the set of all volume-preserving, $\bar{\psi}\psi$-preserving transformations in a 2D vector space), we must add three new vector fields. However, these cannot be the observed weak-interaction bosons, the $W^{+}, W^{-} and Z$: as mentioned before, the new fields must be massless, as adding a mass term for these bosons would violate the local gauge symmetry. How, then, can the masses of the weak bosons fit into the Standard Model? The answer lies in the introduction of yet another field, called the Higgs field, the particle excitation of which is the much-lauded Higgs boson.

\section{The Higgs Mechanism and Electroweak Symmetry Breaking} \label{sec:EWSB} 

By now, it should be clear that the Higgs field is important not simply because it "gives particles mass" (an often-made claim which is, in a sense, true), but because it is a vital missing piece of the Standard Model that is necessary to reconcile the elegant mathematical language of the fundamental interactions with the particles we observe in real-world experiments.

To understand the Higgs mechanism, we must first devote a brief detour to the concept of symmetry breaking. This occurs when an unstable, symmetric state spontaneously changes into a more stable, symmetric one. Consider, for example, the "wine-bottle" potential shape depicted in figure \ref{fig:potential}.

\begin{figure}
\end{figure}

When the ball is balanced at the top of the potential "hill", the configuration is spatially symmetric: that is, no direction is privileged over any other. However, when this delicate balance is even slightly disturbed, the ball will roll down the hill in some direction, resulting in a final state that is \emph{not} symmetrical. This is the phenomenon known as "spontaneous symmetry breaking".

Using the language of quantum field theory, we can easily add spontaneously-breakable symmetric potential terms to a Lagrangian. If we do so, the phenomenon of symmetry breaking allows us to rewrite these terms as a combination of massless fields (one for each "choice" the symmetry breaking must make) and massive fields (corresponding to the leftover degrees of freedom). If we rewrite the "wine-bottle" potential in this way, for example, the one new massless field corresponds to the ball's angular position along the circle at the base, while the massive field corresponds to the ball's  radial position "up" or "down" the hill. The massless particles that arise from these massless fields are known as Goldstone bosons.

We are now ready to discuss the Higgs mechanism. We write the terms of the Standard Model Lagrangian, noting that, since the weak interaction is observed to be chiral, it may couple differently to left- and right-handed fermions.

We can combine the electromagnetic and weak nuclear forces as different manifestations of the same force, called the electroweak force, that transforms like $U(1) \cross SU(2)$.

In this case, the generator of the U(1) symmetry ($frac{\alpha}(x)$) is not the electric charge $Q$ as in our previous example, but is instead the "Weak Hypercharge" $Y = 2(Q -I_{3})$, where $I_{3}$ is the "Weak Isospin". For right-handed particles, $I_{3} = 0$, for left-handed up/charm/top quarks and neutrinos, $I_{3} = +frac{1}{2}$, and for left-handed down/strange/bottom quarks and charged leptons, $I_{3} = +frac{1}{2}$.






\section{The Higgs Boson and Its Couplings} \label{sec:example_subsubsection} 

This is a subsubsection. It will not be included in the table of contents. 

\section{CP-Symmetry}
We return to the Dirac field to discuss its properties. 
