\section{The Standard Model} \label{sec:SM} 

The Standard Model of Particle Physics is arguably one of the crowning achievements of the last century of physics research. Though it is doubtless incomplete (it does not, for instance, explain the dark energy or dark matter observed in cosmological experiments, nor does it provide a satisfactory quantum-mechanical model of gravity), all predictions it has made have yet to be falsified \ref{Peskin}, \ref{Gordy}, \ref{Griffiths}.

The Standard Model is a quantum field theory, meaning that it describes the behavior of fields (physical quantities that are defined at all points in spacetime; common examples of fields include electric and magnetic fields) and their discrete, quantized excitations, referred to as particles (common examples of particles are the electron and the photon, which are excitations of the "electron field" and the electromagnetic field respectively). More about the mathematics of field dynamics will be discussed in section \ref{sec:Lagrangians}.

The Standard Model divides these fundamental particles (and their corresponding fields) into two major categories: the bosons and the fermions. Fermions carry half-integer spin, while bosons carry integer spin. In a general sense, the elementary fermions can be thought of as the particles that comprise matter, while the elementary bosons can be thought of as the particles that correspond to the behavior of the four fundamental forces (electromagnetism, the strong nuclear force, the weak nuclear force, and gravity).

The elementary fermions of the Standard Model all have spin $1/2$. They are divided into two major subgroupings, quarks and leptons; each of these is further divided into three generations. Each generation of quarks contains one up-type quark (up, charm, or top) and one down-type quark (down, strange, and bottom), while each generation of leptons contains one charged lepton (electron, muon, or tau) and one neutrino (electron neutrino, muon neutrino, or tau neutrino). Fermions of successive generations behave similarly to each other, though those of each subsequent generation are more massive than the last.

The quarks are the only fermions that undergo both the strong and weak nuclear interactions, and are also the only particles in the Standard Model that have fractional electromagnetic charge ($+2/3$ for up-type quarks and $-1/3$ for down-type quarks). However, because quarks are never found in isolation and are always bound to other quarks in composite states called hadrons, all observable quark final states have integer charge. Conversely, the charged leptons all have an electromagnetic charge of $-1$, while neutrinos are chargeless. While the quarks and charged leptons have precisely-measured masses, the mass of the neutrinos is vanishingly small, and, to date, only the differences between the different neutrino masses have been conclusively measured.

In addition to these 12 fermion species, each fermion species has a corresponding antimatter "antifermion" species, which has the opposite charge and parity quantum numbers. This is discussed at length in section \ref{sec:CPT}.

The bosons of the standard model are the "messengers" through which the fundamental forces operate (with the exception of the Higgs boson, which is detailed at length in section \ref{sec:EWSB}). The photon and the gluons are massless, while the two W-bosons, the Z-boson, and the Higgs boson are all massive. The photon is the mediator of the electromagnetic force and couples to charged particles; the gluon is the mediator of the strong nuclear force and couples to particles with "color charge", and the W and Z bosons are the mediators of the weak nuclear interaction, coupling to all fermions with left-handed parity and antifermions with right-handed parity. In addition, the three weak bosons ($W^{+}$, $W^{-}$, and $Z$) can all couple to each other, as can the gluons.

The strong nuclear interaction binds quarks together, while the weak interaction governs the decay of one species of fermion into another. Weak interactions operate primarily on fermion doublets, coupling each up-type quark to its corresponding down-type quark and each charged lepton to its corresponding neutrino. However, some intergenerational coupling does occur, the rarity of which is governed by a matrix of coefficients called the Cabibbo-Kobayashi-Makasawa (or CKM) matrix. 

The Higgs boson is the only scalar (spin-0) boson in the Standard Model. Though it does not mediate any force directly, its existence is a consequence of the unification of the electromagnetic and weak nuclear interactions into one "electroweak" interaction at high energy scales. Without the role of the Higgs boson in this process, the fermions, the W-bosons, and the Z-boson would all be massless; thus, the Higgs can be said to "give mass" to the particles of the Standard Model. It couples to all massive particles in the Standard Model, namely, the fermions and the W and Z bosons.

\begin{figure}
\includegraphics[width=\linewidth]{figures/theory_chapter/SM.png}
\caption{The "periodic table" of the Standard Model, depicting the three generations of fermions, the gauge bosons, and the Higgs. Taken from reference \ref{Wikipedia}.}
\end{figure}

\subsection{Lagrangians, Fields, and Gauge Transformations}\label{sec:Lagrangians}

In order to fully explain the Higgs mechanism, we must first discuss the mathematical language of quantum field theory. Both quantum and classical field theories are often discussed using Lagrangian dynamics, where the Lagrangian is defined as  $\mathcal{L} = T - V$, the difference of the kinetic and the potential energy. Physical properties will always evolve in such a way that the integral of this property with respect to time, $\mathcal{S} = \int \mathcal{L} dt$, called the action, is a constant. Lagrangians are also incredibly useful in that they give rise to conservation laws: Noether's Theorem states that operations performed on a system that do not change the Lagrangian are each associated to conserved quantities of a system (i.e., systems with translationally invariant Lagrangians must respect conservation of momentum, systems with temporally-invariant Lagrangians must respect conservation of energy, etc.).

A variety of types of fields exist, but we will discuss three at length: Klein-Gordon fields, which are scalar fields (single quantities defined everywhere), Dirac spinor fields (vector quantities defined everywhere), and gauge fields (additional vector fields that must be introduced in order to preserve certain physical symmetries. We begin with the Klein-Gordon field.

Klein-Gordon fields, one of the simplest examples of a field, are real scalar-valued quantities often denoted using the symbol $\phi$. A non-interacting "free" Klein-Gordon field evolves according to the Lagrangian:

\begin{flalign}
  \begin{aligned}
\mathcal{L} = T - V \\
=\frac{1}{2} \frac{\partial^{2} \phi}{\partial t^{2}} - \frac{1}{2} (\nabla \phi)^{2} - \frac{1}{2} m^{2} \phi^{2} \\
=-\frac{1}{2}(\partial^{\mu}\phi)(\partial_{\mu}\phi)- \frac{1}{2} m^{2} \phi^{2}
  \end{aligned}&&&
\end{flalign}

where we utilize Einstein sum notation in the last line to compress the four derivatives in the preceding expression into a single shorthand symbol. The Lagrangian of an interacting Klein-Gordon field would look similar, but would possess additional $V$ potential terms depending on the nature of the interaction. As the only scalar in the Standard Model, the observable Higgs boson is the only elementary particle to follow an interacting Klein-Gordon field equation.

While the concept of a complex-valued scalar field does not directly correspond to any of the physical elementary particles that are present in the Standard Model, it plays an important role in the understanding of the Higgs mechanism. A complex-valued scalar field behaves similarly to a real-valued one, with the Lagrangian:

\begin{equation}
\mathcal{L} = (\partial^{\mu}\phi^{*})(\partial_{\mu}\phi) - \frac{1}{2} m^{2} \phi^{*} \phi
\end{equation}

Where * denotes the complex conjugate of the scalar field.

Finally, Dirac fields describe all Standard Model fermions (with the possible exception of neutrinos). They are vector fields as opposed to scalar fields, and behave according to the Lagrangian:

\begin{equation}
\mathcal{L} = \bar{\psi}(i \gamma^{\mu} \partial_{\mu} - m) \psi
\end{equation}

where $\gamma^{\mu}$ denotes the set of Dirac gamma matrices, and $\bar{\psi} = \psi^{\dag} \gamma^0$ denotes the transpose of the complex conjugate of the vector field multiplied by one of these matrices, defined as such in order to preserve invariance under relativistic Lorentz boosts. 

A four-component Dirac vector field can be written in a variety of representations, but one of the most useful is that of a doublet of two two-component Weyl spinors, one left-handed and one right-handed, that is, $\psi = \begin{pmatrix} \psi_L \\ \psi_R \end{pmatrix}$. Each of these components transforms differently under Lorentz boosts.

In order to discuss the gauge fields corresponding to the Standard Model bosons, we must first discuss the concept of gauge symmetries. Consider a single Dirac vector field described by equation \ref{eq:Dirac}. We transform the field by rotating it by a constant phase $\alpha$:  $\psi \rightarrow e^{i \alpha} \psi $, $\bar{\psi} \rightarrow e^{-i \alpha} \bar{\psi} $.

\begin{flalign}
  \begin{aligned}
\mathcal{L} = - (e^{-i \alpha} \bar{\psi})(i \gamma^{\mu} \partial_{\mu} - m) (e^{i \alpha} \psi) \\
= - (e^{-i \alpha} e^{i \alpha})\bar{\psi} (i \gamma^{\mu} \partial_{\mu} - m)\psi \\
= -\bar{\psi}(i \gamma^{\mu} \partial_{\mu} - m) \psi 
  \end{aligned}&&&
\end{flalign}

We see that we recover the original Dirac Lagrangian. Thus, the lagrangian is invariant under a constant phase rotation. A transformation of this charater is called a global gauge transformation. A one-dimensional rotation is a unitary transformation, so we call this a global $U(1)$ gauge symmetry.

We next consider the concept of a local gauge transformation, that is, one in which the phase $\alpha$ may vary with position. The field transforms, as before, like  $\psi \rightarrow e^{i \alpha (x)} \psi $, $\bar{\psi} \rightarrow e^{-i \alpha (x)} \bar{\psi} $. However, we note that the dependence on position means that we can no longer factor out the exponentials, and we thus have

\begin{flalign}
  \begin{aligned}
\mathcal{L} = - (e^{-i \alpha (x)} \bar{\psi})(i \gamma^{\mu} \partial_{\mu} - m) (e^{i \alpha (x)} \psi) \\
=-i e^{-i \alpha (x)} \bar{\psi} \gamma^{\mu} (e^{i \alpha (x)} \partial_{\mu} \psi  + i e^{i \alpha (x)} \psi \partial_{\mu} \alpha -m \bar{\psi}\psi \\
=-\bar{\psi}(i \gamma^{\mu} \partial_{\mu} + \gamma^{\mu} \partial_{\mu} \alpha - m) \psi
  \end{aligned}&&&
\end{flalign}

i.e., the Lagrangian is not invariant under this sort of transformation. However, local gauge invariance is an important physical symmetry, so in order to attempt to preserve it, we add an additional term to the original Lagrangian: an extra vector field $A_\mu$ that transforms like $A_\mu \rightarrow A_\mu - \frac{1}{q} \partial_{\mu} \alpha(x) $ for a constant q. We also define a new operator, called the covariant derivative: $D_{\mu} = \partial_{\mu}+ iqA_{\mu} $. Given how $A_{\mu}$ transforms, we see that $D_{\mu}$ transforms like

\begin{flalign}
  \begin{aligned}
D_{\mu} = \partial_\mu + iqA_{\mu} \\
\rightarrow \partial_\mu + iq(A_{\mu}-\frac{1}{q}\partial_{\mu}\alpha(x)) \\
= \partial_\mu + iqA_{\mu}-i\partial_{\mu}\alpha(x) \\
=D_{\mu}-i\partial_{\mu}\alpha(x)
  \end{aligned}&&&
\end{flalign}

Let us replace the $\partial_{\mu}\phi$ terms in the initial Lagrangian with $D_{\mu}\phi$ and transform.

\begin{flalign}
  \begin{aligned}
\mathcal{L} = - (e^{-i \alpha (x)} \bar{\psi})(i \gamma^{\mu} (D_{\mu}-i\partial_{\mu}\alpha(x)) - m) (e^{i \alpha (x)} \psi) \\
=-(e^{-i \alpha (x)} \bar{\psi})(i \gamma^{\mu} (\partial_\mu + iqA_{\mu}-i\partial_{\mu}\alpha(x))-m)(e^{i \alpha (x)} \psi) \\
=-\bar{\psi}(i \gamma^{\mu} D_{\mu} + \gamma^{\mu} \partial_{\mu} \alpha - \gamma^{\mu} \partial_{\mu} \alpha - m) \psi \\
=-\bar{\psi}(i \gamma^{\mu} D_{\mu} - m) \psi
  \end{aligned}&&&
\end{flalign}

Thus, by introducing an additional vector field that corresponds to the local U(1) gauge symmetry, we have restored the invariance of our lagrangian. Physically, this field is analogous to the introduction of electromagnetism to our single Dirac fermion model, with the $A_\mu$ field playing the role of the photon: it is a vector quantity and so has spin-1, it must be massless (as adding an $A_\mu$ mass term to the Lagrangian would violate the symmetry again), and couples to the fermion fields according to their electromagnetic charge. We note that we can also still preserve invariance if we add an additional term $L_{kin} = -\frac{1}{4}F^{\mu \nu} F_{\mu \nu}$ to the Lagrangian, where $F_{\mu \nu} = \partial_{\mu} A_{\nu} - \partial_{\nu} A_{\mu}$: this corresponds to the energy inherent in electromagnetic fields themselves. 

Each of the fundamental forces in the Standard Model can be understood in terms of these sorts of gauge symmetries. The photon is the particle excitiation of the electromagnetic field, which corresponds to a one-dimensional rotation "U(1)" transformation. In order to be invariant under three dimensional transformations of the type "SU(3)" (the set of all volume-preserving, $\bar{\psi}\psi$-preserving transformations in a 3D vector space), we must add eight new vector fields (these are the eight gluons), leading to the incorporation of the strong interaction into the Standard Model.

Similarly, in order to be invariant under two-dimensional transformations of the type "SU(2)" (the set of all volume-preserving, $\bar{\psi}\psi$-preserving transformations in a 2D vector space), we must add three new vector fields. However, these cannot be the observed weak-interaction bosons, the $W^{+}, W^{-} and Z$: as mentioned before, the new fields must be massless, as adding a mass term for these bosons would violate the local gauge symmetry. How, then, can the masses of the weak bosons fit into the Standard Model? The answer lies in the introduction of yet another field, called the Higgs field, the particle excitation of which is the much-lauded Higgs boson.

\section{CP-Symmetry}\label{sec:CP}

In addition to the $SU(2)_{L} \times U(1) \times SU(3)$ gauge symmetry, the Standard Model also respects a discrete symmetry called CPT. This is the combined product of three separate symmetries: charge conjugation (C), parity inversion (P), and time reversal (T). While some interactions may respect a limited combination of these symmetries (i.e., just C, or the product CP), all Standard Model processes must respect the product of the three: that is, for a given physics process, if we invert all the charges, flip all the parities, and run the interaction backward, the resulting new process must also be physically allowed.

The strong nuclear, electromagnetic, and gravitational interactions are symmetric under C, P, and T individually. The weak nuclear interaction, however, is invariant only under the combination of all three: it violates C, P, and, in some cases, CP. The CP violation that occurs in the weak interaction is through the CKM off-diagonal quark-mixing matrix mentioned briefly in section \ref{sec:SM}.

A parity inversion is equivalent to reversing a particle's momentum without reversing its spin. For a fermion field $\psi(t,x)$, a parity transform P takes $\psi(t,x)$ to $\psi(t,-x)$, and looks like $P \psi(t,x) P = e^{i \theta} \gamma^{0} \psi(t,-x)$ for some constant matrix $\theta$. 

Given this, we find that the expression $P \bar{\psi}\psi P = \bar{\psi}\psi$ , so $\bar{\psi}\psi$ is a $\emph{scalar}$ under parity, while the expression $iP \bar{\psi}\gamma^{5}\psi P = -i\bar{\psi}\gamma^{5}\psi$ acquires a minus sign, so $i\bar{\psi}\gamma^{5}\psi$ is a $\emph{pseudoscalar}$ under parity.

A time-reversal is equivalent to "running a process backward". For example, a swinging pendulum is a process that is approximately T symmetric, as it looks similar "played" in forward or reverse, while a plate shattering on the ground is not a T symmetric process, as reversing it does not make physical sense (broken plates do not spontaneously reform). For a fermion field $\psi(t,x)$, a time-reversal transform T takes $\psi(t,x)$ to $\psi(-t,x)$, and looks like $T \psi(t,x) T = -\gamma^{1}\gamma^{3} \psi(-t,x)$. Given this, we find that the expression $T \bar{\psi}\psi T = \bar{\psi}\psi$ , so $\bar{\psi}\psi$ is a $\emph{scalar}$ under parity, while the expression $iT \bar{\psi}\gamma^{5}\psi T = -i\bar{\psi}\gamma^{5}\psi$ acquires a minus sign, so $i\bar{\psi}\gamma^{5}\psi$ is a pseudoscalar under time-reversal.

A charge conjugation is a reversal of all charges (electric charge, weak hypercharge, and strong-nuclear color charge). For a fermion field $\psi(t,x)$, a charge conjugation transform C takes $\psi$ to $\bar{psi}$, and looks like $C \psi C = -i (\bar{\psi} \gamma^{0} \gamma^{2})^{T}$ (where T here denotes the matrix transpose). Given this, we find that the expression $C \bar{\psi}\psi C = \bar{\psi}\psi$ , so $\bar{\psi}\psi$ is a $\emph{scalar}$ under charge conjugtion, as is the expression $iC \bar{\psi}\gamma^{5}\psi C = i\bar{\psi}\gamma^{5}\psi$ is as well. Thus, Lagrangians containing terms of the form $i\bar{\psi}\gamma^{5}\psi$ are invariant under C and CPT, but not P, T or CP.

Applying C and P together is equivalent to switching particles and antiparticles. However, we also know that our universe is composed almost entirely of matter, and not equal parts matter and antimatter (living in a universe in which matter and antimatter are equally abundant would be very unpleasant, as the two would be constantly annhilating each other, bathing us in a near-constant shower of gamma rays). Thus, some physics process that occurs in the high-energy regime near the Big Bang must violate CP in a substantial way. The CP-invariance that occurs through the CKM mixing is not enough to account for the matter-antimatter asymmetry, so searching for this source of CP violation is of pressing interest to experimental physicists.

\section{The Higgs Mechanism and Electroweak Symmetry Breaking} \label{sec:EWSB} 

Given the results of previous sections, it should be clear that the Higgs field is important not simply because it "gives particles mass" (an often-made claim which is, in a sense, true), but because it is a vital missing piece of the Standard Model that is necessary to reconcile the elegant mathematical language of the fundamental interactions with the particles we observe in real-world experiments.

To understand the Higgs mechanism, we must first devote a brief detour to the concept of spontaneous symmetry breaking. This occurs when an unstable, continuous symmetric state spontaneously changes into a more stable, asymmetric one. Consider, for example, the "wine-bottle" potential shape depicted in figure \ref{fig:potential}.

\begin{figure}
\includegraphics[width=\linewidth]{figures/theory_chapter/higgspotential.png}
\caption{The "wine bottle" or "Mexican hat" potential hill, from reference \ref{Ellis}}
\end{figure}

When the ball is balanced at the top of the potential "hill", the configuration is spatially symmetric: that is, no direction is privileged over any other. However, when this delicate balance is even slightly disturbed, the ball will roll down the hill in some direction, resulting in a final state that is $\emph{not}$ symmetrical. This is the phenomenon known as "spontaneous symmetry breaking".

Using the language of quantum field theory, we can easily add spontaneously-breakable symmetric potential terms to a Lagrangian. If we do so, the phenomenon of symmetry breaking allows us to rewrite these terms as a combination of massless fields (one for each "choice" the symmetry breaking must make) and massive fields (corresponding to the leftover degrees of freedom). If we rewrite the "wine-bottle" potential in this way, for example, the one new massless field corresponds to the ball's angular position along the circle at the base, while the massive field corresponds to the ball's  radial position "up" or "down" the hill. The massless particles that arise from these massless fields are known as Goldstone bosons.

We are now ready to discuss the Higgs mechanism. We can combine the electromagnetic and weak nuclear forces as different manifestations of the same force, called the electroweak force, that transforms like $U(1) \times SU(2)$. We write the terms of the Standard Model electroweak Lagrangian, noting that, since the weak interaction is observed to be chiral, it couples differently to left- and right-handed fermions.

In this case, the generator of the U(1) symmetry ($\alpha(x)$) is not the electric charge $Q$ as in our previous example, but is instead the "Weak Hypercharge" $Y = 2(Q -I_{3})$, where $I_{3}$ is the third component of the "Weak Isospin" $I$. For right-handed particles, $I_{3} = 0$, for left-handed up/charm/top quarks and neutrinos, $I_{3} = +\frac{1}{2}$, and for left-handed down/strange/bottom quarks and charged leptons, $I_{3} = +\frac{1}{2}$. We will call our U(1) boson $B_{\mu}$ rather than $A_{\mu}$, to distinguish it from the photon. For a fermion $\psi$ (u/c/s/e/$\mu$/$\tau$) that couples weakly to another fermion $\psi'$, we can write the left-and-right-handed components of f as 

$L = \begin{pmatrix} \psi_{L} \\ \psi'_{L} \end{pmatrix}$, $R = \begin{pmatrix}\psi_{R} \end{pmatrix}$.

Similarly, the generators of the SU(2) symmetry are the three matrices $\tau_{i}$, one corresponding to each weak boson $W^{\mu}_{i}$. The self-interaction term for the weak interaction is $L_{kin} = -\frac{1}{4}G^{\mu \nu}G_{\mu \nu}$, where $G_{\mu \nu} = \partial_{\mu} W_{\nu} - \partial_{\nu} W_{\mu}+\frac{g}{2} W_{\mu}\times W_{\nu}$.

We can define a new covariant derivative capturing the weak interaction terms as well: $D^{\mu} = \partial^{\mu}- \frac{1}{2}iYg_{1}B^{\mu} -\frac{1}{2}i I g_2 \Sigma \tau_{i}W_{i}^\mu$ Thus, the full electroweak $SU(2)_L \times U(1)$ Lagrangian is:

\begin{equation}
\mathcal{L} = i\bar{R}\gamma^{\mu}D_{\mu}R + i\bar{L}\gamma^{\mu}D_{\mu}L -\frac{1}{4}G^{\mu \nu}G_{\mu \nu} -\frac{1}{4}F^{\mu \nu} F_{\mu \nu}
\end{equation}

However, we note that this Lagrangian does not allow for gauge boson mass or fermion mass, as a fermion mass term $m\bar{\psi}\psi = m(\bar{\psi_{R}}\psi_{L}+\bar{\psi_{L}}\psi_{R})$ is not an SU(2) singlet and would thus not be invariant under the SU(2) symmetry. 

To rectify this, we can introduce a new complex SU(2) doublet scalar field $\phi = \begin{pmatrix} \phi^{+} \\ \phi^{0} \end{pmatrix} = \begin{pmatrix} \phi_{1} + i\phi_{2}  \\ \phi_{3} + i\phi_{4} \end{pmatrix}$ called the $\emph{Higgs field}$. It follows the Lagrangian

\begin{equation}
\mathcal{L} = T-V = (D^{\mu}\phi)^{\dag}(D_{\mu}\phi) - {\mu}^2\phi^{\dag}\phi - \lambda(\phi^{\dag}\phi)^{2}
\end{equation}

which, if $mu^{2} < 0$ has a minimum at
\begin{equation}
\phi^{\dag}\phi = \frac{-\mu^{2}}{2 \lambda} = \frac{v^{2}}{2}
\end{equation}

We note that the potential is symmetric, so we can pick a direction to "break the symmetry" in. Let us define the vacuum as $\phi_0 = \frac{1}{\sqrt{2}} \begin{pmatrix} 0 \\ v \end{pmatrix}$. Then, for any arbitrary configuration of the field, we can expand the potential about the minimum as $\phi(x) = \frac{1}{\sqrt{2}} \begin{pmatrix} 0 \\ (v + h(x)) \end{pmatrix}$.

We can expand the kinetic term in the Lagrangian as

\begin{equation}
(D^{\mu}\phi)^{\dag}(D_{\mu}\phi) = \frac{1}{2}(\partial_{\mu}h)(\partial^{\mu}h)+ \phi^{\dag}(\frac{1}{2}iYg_{1}B_{\mu} -\frac{1}{2}iI g_2 \Sigma \tau_{i}W_{i\mu})^{\dag}(\frac{1}{2}iYg_{1}B^{\mu} -\frac{1}{2}iI g_2 \Sigma \tau_{i}W_{i}^mu) \phi
\end{equation} 

Which, using the definitions of the three tau matrices, and putting Y = I = 1, becomes

\begin{flalign}
  \begin{aligned}
\frac{1}{2}(\partial_{\mu}h)(\partial^{\mu}h) + \frac{1}{8} |\begin{bmatrix} g_{1}B_{\mu}+g_{2}W_{\mu}^{3} & g_{2}(W_{\mu}^{1}-iW_{\mu}^{2}) \\ g_{2}(W_{\mu}^{1}+iW_{\mu}^{2}) & g_{1}B_{\mu}+g_{2}W_{\mu}^{3} \end{bmatrix} \begin{bmatrix} 0 \\ v \end{bmatrix}|^{2} \\
= \frac{1}{2}(\partial_{\mu}h)(\partial^{\mu}h) + \frac{1}{8}v^{2}g_{2}^{2}((W_{\mu}^{1})^{2}+(W_{\mu}^{2})^{2})) + \frac{1}{8}v^{2}(g_{1} B{\mu}-g_{2}W_{\mu}^{3})^{2}
  \end{aligned}&&&
\end{flalign}

If we again define some new field combinations here: $W_{\mu}^{+} -\frac{1}{\sqrt{2}}(W_{\mu}^{1}-iW_{\mu}^{2})$, $W_{\mu}^{-} -\frac{1}{\sqrt{2}}(W_{\mu}^{1}+iW_{\mu}^{2})$, $Z_{\mu} = \frac{g_{1}B{\mu}+g_{2}W_{\mu}}{\sqrt{{g_1}^2+{g_2}^2}}$ and $A_{\mu} = \frac{g_{1}B{\mu}-g_{2}W_{\mu}}{\sqrt{{g_1}^2+{g_2}^2}}$,

then we get: 
\begin{equation}
\frac{1}{2}(\partial_{\mu}h)(\partial^{\mu}h) + (\frac{1}{2}vg_{2})^{2}W_\mu^{+}W_\mu^{-}+(\frac{1}{2}v(\sqrt{{g_1}^2+{g_2}^2}))^{2}Z_{\mu}
\end{equation}

And we see that Z field and the two W fields have acquired mass terms, while the A field has remained massless. This corresponds to what we observe in the real world: the W bosons and the Z boson are massive, while the photon remains massless!

For fermions, the coupling to the Higgs field looks like $g_{f}(\bar{f}_{L}\phi f_{R} + \phi^{\dag}\bar{f}_{R} f_{L})$ (for some constant $g_{f}$).

We note that, if we expand $\phi(x) = \frac{1}{\sqrt{2}} \begin{pmatrix} 0 \\ (v + h(x)) \end{pmatrix}$ as before, we find that the coupling term now looks like 
$\frac{g_{f}v}{\sqrt{2}}(\bar{f}_{L}f_{R} + \bar{f}_{R}f_{L}) + \frac{g_{f}}{\sqrt{2}}(\bar{f}_{L}f_{R} + \bar{f}_{R}f_{L})$. Thus, the presence of the Higgs field allows us to have fermion mass terms in our Lagrangians, where the masses are $m_{f} = \frac{g_{f}v}{\sqrt{2}}$. However, we note that, at the current time, there is no way to derive the $g_{f}$ coupling theoretically: thus, while the Higgs mechanism now allows us to have massive fermions in our theory, it does not actually determine what the masses of these fermions are.

Finally, we note that if we expand the potential term in the complex Klein-Gordon Lagrangian for the Higgs field about the minimum, that is, 

\begin{flalign}
  \begin{aligned}
V = {\mu}^2\phi^{\dag}\phi + \lambda(\phi^{\dag}\phi)^{2} = \frac{1}{2}{\mu}^2(v+h)^{2} + \frac{1}{4}\lambda(v+h)^{4} \\
= -\frac{1}{2}{\lambda v^{2}}(v+h)^{2} + \frac{1}{4}\lambda(v+h)^{4} \\
= -\frac{1}{2}\lambda(v^{4} + 2v^{3}h + v^{2}h^{2}) + \frac{1}{4}\lambda(v^{4}+4vh^{3}+6v^{2}h^{2} + 4v^{3}h + h^{4})
  \end{aligned}&&&
\end{flalign}

we see a mass term of the form $(\frac{M_{h}^{2}}{2})h^{2} = \frac{6}{4}\lambda v^{2} h^{2} - \frac{1}{2}\lambda v^{2} h^{2} = \lambda v^{2} h^{2}$;
That is, $M_{h} = v \sqrt{2 \lambda}$.

In sum: without a Higgs field, in order for the Standard Model electroweak Lagrangian to preserve $SU(2)_L \times U(1)$ symmetry, we are forbidden from including mass terms for the fermions or the gauge bosons. However, by adding this additional field to the model and breaking its symmetry, we find that we can include fermion mass terms that do not violate the $SU(2)_L \times U(1)$ symmetry, and that the electroweak gauge bosons now acquire mass terms by default. Furthermore, this new field also has a mass. This is a remarkable result: by simply adding one extra field to the model, we have solved several different problems in one fell swoop!

\section{The Higgs Boson and Its Couplings} \label{sec:Higgs Couplings} 

As we have seen, the Standard Model Higgs is massive, and couples directly to all other massive particles (with the possible exception of neutrinos; this is an active area of research). However, the existence of the Higgs is clearly not all there is to the story: the Standard Model is noticeably incomplete (What are dark matter and dark energy? What determines the exact masses of each of the fermions? Why have we never observed right-handed neutrinos? Can we unify other forces like we have unified electromagnetism and the weak force?). In the majority of proposed Beyond-the-Standard Model extensions, other particles that would interact with the Standard Model Higgs are introduced, including additional Higgs fields, as well as new massive fermions and bosons.

With the discovery of the Higgs, it can be argued that we have now entered an era of precision-measurement in collider physics. By closely studying interactions that are very sensitive to the injection of new physics, we can potentially detect deviations from the Standard Model: in many cases, we may see hints of the existence of new particles through changes to quantities of interest well before we are actually able to produce these particles in colliders directly. Due to the Higgs coupling to all massive particles, Higgs interactions are especially sensitive to new physics, and thus can serve as a useful bellwether for physics beyond the standard model.

The Large Hadron Collider is a proton-proton collider; as such, physics processes that occur in it are almost always initiated by quarks or gluons. The predominant production modes are, in order of frequency, gluon-gluon fusion (ggF), vector boson fusion (VBF), vector-boson 
associated (VH) and top-associated (ttH). The top-associated Higgs production mode is of particular interest to many experimentalists due to its direct dependence on the Higgs coupling to a fermion- by closely measuring the frequency and properties of the ttH mode, we can better understand the nature of this variety of couplings. Rarer production modes, such as single-top associated Higgs production (tH), can shed still more light on the nature of the top-quark/Higgs coupling (called the top quark Yukawa coupling), but are much more difficult to observe.

\begin{figure}[htp]
  \centering
       \begin{subfigure}[b]{0.3\textwidth}
         \centering
         \includegraphics[width=\textwidth]{figures/theory_chapter/VBF.png}
         \caption{VBF}
         \label{fig:VBF}
     \end{subfigure}
     \hfill
         \begin{subfigure}[b]{0.3\textwidth}
         \centering
         \includegraphics[width=\textwidth]{figures/theory_chapter/ggF.png}
         \caption{ggF}
         \label{fig:ggH}
     \end{subfigure}
     \hfill 
     \\
        \begin{subfigure}[b]{0.3\textwidth}
         \centering
         \includegraphics[width=\textwidth]{figures/theory_chapter/VH.png}
         \caption{VH}
         \label{fig:VH}
     \end{subfigure}
     \hfill
         \begin{subfigure}[b]{0.3\textwidth}
         \centering
         \includegraphics[width=\textwidth]{figures/theory_chapter/ttH.png}
         \caption{ttH}
         \label{fig:ttH}
     \end{subfigure}
     \hfill 
  \label{fig:Higgsmodes}
  \caption{Feynman diagrams depicting the leading Higgs production modes.}  
\end{figure}

\begin{figure}[htp]
  \centering
       \begin{subfigure}[b]{0.3\textwidth}
         \centering
         \includegraphics[width=\textwidth]{figures/theory_chapter/tWH1.png}
         \caption{A tWH mode}
         \label{fig:tWH1}
     \end{subfigure}
     \hfill
         \begin{subfigure}[b]{0.3\textwidth}
         \centering
         \includegraphics[width=\textwidth]{figures/theory_chapter/tWH2.png}
         \caption{A tWH mode}
         \label{fig:tWH2}
     \end{subfigure}
     \hfill 
     \\
        \begin{subfigure}[b]{0.3\textwidth}
         \centering
         \includegraphics[width=\textwidth]{figures/theory_chapter/tqH1.png}
         \caption{A tqH mode}
         \label{fig:tqH1}
     \end{subfigure}
     \hfill
         \begin{subfigure}[b]{0.3\textwidth}
         \centering
         \includegraphics[width=\textwidth]{figures/theory_chapter/tqH2.png}
         \caption{A tqH mode}
         \label{fig:tqH2}
     \end{subfigure}
     \hfill 
  \label{fig:tHmodes}
  \caption{Feynman diagrams depicting single-top associated Higgs production modes.}  
\end{figure}

\begin{figure}[htp]
  \centering
       \begin{subfigure}[b]{0.3\textwidth}
         \centering
         \includegraphics[width=\textwidth]{figures/theory_chapter/toploop.png}
         \caption{Top-mediated}
         \label{fig:tWH1}
     \end{subfigure}
     \hfill
         \begin{subfigure}[b]{0.3\textwidth}
         \centering
         \includegraphics[width=\textwidth]{figures/theory_chapter/Wloop1.png}
         \caption{W $\rightarrow \gamma$ }
         \label{fig:tWH2}
     \end{subfigure}
     \hfill 
        \begin{subfigure}[b]{0.3\textwidth}
         \centering
         \includegraphics[width=\textwidth]{figures/theory_chapter/Wloop2.png}
         \caption{W $\rightarrow \gamma \gamma$}
         \label{fig:tqH1}
     \end{subfigure}
     \hfill
  \label{fig:loopmodes}
  \caption{Feynman diagrams depicting the leading-order processes contributing to the Higgs diphoton decay.}  
\end{figure}

Similarly, the dominant decay mode of the Higgs is to bottom quarks ($H \rightarrow bb$), followed by to W bosons ($H \rightarrow WW$), gluons ($H \rightarrow WW$), tau leptons ($H \rightarrow \tau \tau$), charm quarks ($H \rightarrow bb$), Z bosons ($H \rightarrow ZZ$), photons ($H \rightarrow \gamma \gamma$), and a Z boson and a photon ($H \rightarrow Z \gamma$). Because the Higgs does not couple to massless particles directly, the decays to gg and $\gamma \gamma$ are mediated by loop diagrams, most often involving top quarks.

\begin{figure}
\includegraphics[width=\linewidth]{figures/theory_chapter/HiggsBR.png}
\caption{The branching ratio of the Higgs to various final state particles as a function of its mass (now known to be ~125 GeV), from reference \ref{2011WG}.}
\end{figure}

Investigating each of these decay modes has different benefits and detriments. While the Higgs decay to bottom quarks is the most common, correctly identifying and accurately reconstructing quarks and gluons using the jets that they produce in detectors is often very difficult. Decays to W bosons, Z bosons, and $\tau$ leptons provide "cleaner" channels, but because the W, Z and $\tau$ decay dominantly to hadrons, similar reconstruction issues occur unless we restrict ourselves to the rarer leptonic subchannels of these decays. The Higgs to diphoton channel occurs very rarely, but offers a much more unambiguous signal, as the odds of misidentifying high-energy gamma rays are fairly low.

Additionally, we must consider combinatorics: in the ttH production mode, for instance, the final state contains two bottom quarks from the decay of the associated top quarks. If we choose to examine the subchannel in which the Higgs decays to bottom quarks as well, we find that we have at least four final-state bottom quark jets in the event, with 12 different unique assignments of bottom quark jets to parent particles. Correctly matching up which jet originated from which parent particle thus further complicates the reconstruction problem.

This dissertation features two physics analyses from the ATLAS experiment. The first is a targeted attempt to measure whether the top quark Yukawa coupling is CP-even (respects CP symmetry) or CP-odd (violates it) by closely studying the $ttH$ process. In the Standard Model Lagrangian, the top Yukawa coupling is purely CP-even, and can be written in the form $\kappa y_{t} \bar{t}th$. Adding a CP-odd component to the coupling adds an additional $\kappa y_{t} \bar{t} (i \gamma^{5})th$ term to the Lagrangian; the total coupling can thus be parameterized as $\kappa y_{t} \bar{t} (cos(\alpha)+ sin(\alpha) i \gamma^{5} )th$ for a "CP-mixing angle" $\alpha$ that governs the relative magnitude of the CP-even and CP-odd terms. Introducing this CP-odd term changes the kinematics of the process, and leads to events with different angular variable values (such as the angle between the two tops) as well as changing the energies and momenta of some decay products. By performing a fit on these variables to determine the relative agreement or disagreement with the standard model, we can extract a value for $\alpha$.

The second analysis discussed here is a survey of a wide variety of Higgs-based physics processes using the Higgs-to-diphoton decay channel. A combined fit is performed in different regions of phase space that correspond to the ggH, VBF, VH, ttH, and tH processes; by doing this, we can extract limits on the individual Higgs couplings $\kappa$.

