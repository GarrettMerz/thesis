The discovery of the Higgs boson in 2012 by CERN's ATLAS \cite{HiggsATLAS} and CMS \cite{HiggsCMS} collaborations at the Large Hadron Collider marked a watershed moment in the history of collider physics. With its discovery, the Standard Model of particle physics, developed in the 1970s by scientists attempting to understand the deluge of new particles discovered throughout the 20th century, had finally been completed. However, countless physics questions remained (and still do remain) unanswered - why is the universe comprised of matter rather than antimatter? What are dark matter and dark energy? Can the four fundamental forces be unified?

With the Standard Model unable to provide answers to these questions, physicists have begun to search for clues in precision measurements of existing particles and their properties. Many new physics phenomena may appear as subtle hints in these measurement regimes before their eventual direct observation at the colliders of tomorrow. This has precedent in the field of Higgs physics- though predicted by the Standard Model in the 1970s, indirect evidence of the Higgs was first seen through precision measurements with the LEP experiment almost a decade before its official discovery \cite{LEPExperiments}.

One such avenue for these precision measurements using current experiements is that of the interactions of the Higgs boson with the other particles of the Standard Model. By measuring how strongly the Higgs interacts with these particles and how often its various interactions occur (parameterized mathematically using quantities called 'couplings' and 'cross-sections', respectively), physicists can search for experimental disagreements with the Standard Model and set limits on the allowed parameters of potential beyond-the-Standard-Model physics.

Of particular interest is the Higgs coupling to the top quark, the heaviest of all known fundamental particles. Because the Higgs couplings are intimately related to the origins of the quark and lepton masses, the coupling between the Higgs and the top is of special theoretical importance, as it can in many cases serve as a window on new physics in the Higgs sector. By observing top-quark pair production in association with a Higgs boson ($ttH$), a rare process only recently observed at the LHC \cite{ttH}, this coupling can be measured precisely and its properties ascertained.

The work in this dissertation discusses the measurement of Higgs boson processes in the diphoton decay channel ($H \rightarrow \gamma \gamma$) using proton-proton collisions with a center-of-mass energy of $\sqrt{s} = 13$ TeV, recorded with the ATLAS detector.  Both analyses presented in this dissertation use the full ATLAS Run 2 dataset, which consists of $139 fb^{-1}$ of data gathered between 2015 and 2018. Two analyses are presented- one a precision measurement of the charge-parity (CP) symmetry of the top-quark-Higgs coupling (previously published as \cite{CPanalysis}), and another a measurement of the Higgs production cross-sections for various processes in a number of theoretically-motivated kinematic regimes (previously published as \cite{Couplingsanalysis}).
 
In both analyses, measurements are performed by categorizing events gathered using the ATLAS detector into signal-enriched regions of interest using machine-learning algorithms called Boosted Decision Trees (BDTs). These tools are trained on Monte Carlo simulated data to combine the effects of multiple discriminatory variables, allowing us to focus analysis efforts on regions with a higher signal significance. In these categories, the diphoton mass distribution $m_{\gamma \gamma}$ is then fit with a smoothly-falling background functional form (which models the shape of the QCD continuum diphoton background) and a Gaussian-like "Double-Sided Crystal Ball" signal functional form (which models the shape of the Higgs peak), and the number of fitted signal events in each category is extracted. Using the signal yields in multiple categories, various quantities of interest can be measured.

The structure of this work is as follows: first, the theoretical background and phenomenology of the Standard Model are discussed in Chapter 2. Special attention is paid to the Higgs boson and its couplings, as well as the nature of charge-parity (CP) symmetry. In Chapter 3, an overview of the construction and operation of the ATLAS detector is given. In Chapter 4, the reconstruction of particles and their properties using detector signals is elaborated upon. Chapter 5 discusses the parameters of the various data and Monte Carlo simulation samples used in the analyses. Chapter 6 details the common signal and background parameterization procedures for the two analyses discussed in this dissertation. In Chapter 7, a measurement of the CP properties of the top Yukawa coupling is outlined; the result of this measurement is presented in Chapter 8. In Chapter 9, a measurement of the Higgs boson production cross-section in a number of Simplified Template Cross-Section (STXS) kinematic regions is discussed. In Chapter 10, the conclusion chapter, the results of both analyses are discussed.