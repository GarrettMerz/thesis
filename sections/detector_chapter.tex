\section{The Large Hadron Collider} \label{sec:LHC} 

At roughly 27 kilometers in circumference, CERN's Large Hadron Collider (LHC) is the largest machine ever built, slicing through the solid rock of the Jura mountains and running more than 100 meters beneath the Swiss-French border \ref{LHC}. It is a proton-proton collider operating at a center-of-mass energy of 13 TeV, circulating two beams of protons in opposite directions, each at more than 99\% the speed of light \ref:{speed}. By utilizing a system of radiofrequency (RF) cavities and dipole magnets, protons are delivered to four primary collision points in bunches spaced 25 nanoseconds apart \ref:{40Mhz}.

The LHC achieves such a powerful center-of-mass energy by repurposing a significant fraction of older collider physics infrastructure, including the Super Proton Synchrotron (SPS), which was itself one of the most powerful particle accelerators in the world at one time. In order to achieve collision, hydrogen atoms are first stripped of their electrons using an ionizing cathode filament in a device called a 'Duoplasmatron'; this produces a plasma that is filtered to produce beams of protons \ref{}. The LINAC2, a linear accelerator, then accelerates the resultant proton beams to a collision energy of 50 MeV; the beams then move to the Proton Synchrotron Booster (PSB), a circular accelerator that accelerates them to an energy of 1.4 GeV, followed by the Proton Synchrotron (PS), a second circular accelerator that accelerates them to 25 GeV. Following this, beams enter the aforementioned Super Proton Synchrotron, where they are accelerated to a collision energy of 450 GeV before being injected into the Large Hadron Collider ring. This infrastructure is depicted in Figure \ref{fig:LHC}.

The tight bunching of protons by the LHC ring allows the collider to deliver collisions at a high luminosity, a measure of the number of expected collision events per unit of beam area. Luminosity is measured in both instantaneous and integrated form (i.e., 'luminosity per unit time' and total luminosity delivered to date'); as of the end of the second major LHC run, the collider has delivered a total integrated luminosity of XXX fb^{-1}, equivalent to X.XX*10^{41} collisions per square centimeter \ref{lumi}. 

At the time of this writing, the LHC is nearing the end of 'Long Shutdown 2' (LS2), a two-year long upgrade period designed to increase detector performance and lay the groundwork for the upcoming high-luminosity LHC (HL-LHC) upgrade, which is slated to begin in approximately 2027 and aims to increase the design luminosity of the LHC by at least a factor of 5 \ref:{HL-LHC}. Higher luminosity allows for the production of more rare physics events but also dramatically increases the incidence of unwanted "pileup" events; mitigating this is one of the major efforts of the CERN physics community during the lead-up to the HL-LHC run.

The LHC ring accelerates beams to a center-of-mass collision energy of 13 TeV. The ring consists of two primary beam pipes, one containing a "clockwise" beam and the other containing a "counterclockwise" beam, which overlap at the four primary collision points. These correspond to the four major LHC physics experiments: ATLAS  \ref{} and CMS \ref{}, which are "general purpose" physics detectors, LHCb \ref{}, which is specialized to study the physics of hadrons containing bottom (or "beauty") quarks, and ALICE \ref{}, which is primarily designed for heavy-ion physics.

\section{The ATLAS Detector} \label{sec:ATLAS} 

The work contained in this dissertation was perormed using the ATLAS (A Large Toroidal LHC ApparatuS) detector. The detector is 




\subsection{Inner Detector} \label{sec:ID} 

This is a subsection. It will be included in the table of contents. 

\subsection{Calorimeters} \label{sec:Calos} 

This is a subsection. It will be included in the table of contents. 

\subsubsection{ECAL} \label{sec:ECAL} 

\subsubsection{HCAL} \label{sec:HCAL} 

\subsection{Magnets} \label{sec:Magnets}

\subsubsection{solenoid} \label{sec:solenoid} 

\subsubsection{toroid} \label{sec:toroid} 

\subsection{Muon Spectrometer} \label{sec:Musyst}
 
\subsection{Trigger} \label{sec:Trigger}
