\section{Study of the CP properties of the top-quark Yukawa interaction in t\bar{t}H and tH events wth H \rightarrow \gamma \gamma} \label{sec:ttHCPCategorization}

In order to properly measure the dependence of the top Yukawa coupling on the CP-mixing angle $\alpha$, we divide a region of $ttH$-and-$tH$-enriched phase space into a number of different categories based both upon the similarity of events it contains to signal (Higgs processes like $ttH$ and $tH$) rather than background (non-Higgs continuum diphoton processes), as well as the similarity of events it contains to CP-odd rather than CP-even Higgs processes. By creating many such categories and fitting the event yield in each, we can set detailed constraints on the value of $\alpha$.

First, we define two sets of regions. The "hadronic" region targets events containing fully-hadronic top decays, requiring two loose-ID photons, one b-tagged jet at the 77\% working point with p_{T} > 25 GeV, as well as two additional jets with p_{T} > 25 GeV and exactly zero electrons or muons. Similarly, the "leptonic" region targets events containing semi-leptonic top decays, requiring two loose-ID photons, one b-tagged jet at the 77\% working point with p_{T} > 25 GeV, as well as at least one isolated electron or muon.

To perform the categorization in these regions, two multivariate Boosted Decision Trees (BDTs) are used, one to separate Higgs events from background and one to separate CP-odd Higgs events from CP-even Higgs events. Both BDTs are trained on low-level kinematic features using the XGBoost package \ref{cite:XGBoost}.

\subsection{SBBDT}

The signal-versus-background BDT (SBBDT) developed for use in the CP Analysis is identical to that developed first in the $79.8 fb^{-1}$ measurements of $ttH$ in the diphoton channel \ref{cite:ATLAS-CONF-2019-004} in and later retrained for $139 fb^{-1}$ measurements of $ttH$ in the diphoton channel \ref{cite:ATLAS-CONF-2019-004}. It is trained separately for both the hadronic and leptonic regions.

Both the hadronic and leptonic BDTs are trained using a Standard-Model Powheg $ttH$ Monte Carlo sample to model the signal and NTI data control events to model the continuum diphoton background.

\subsubsection{Hadronic Region} \label{sec:SBBDThad} 
In the hadronic region, 60\% of the $ttH$ Monte Carlo signal events are used for training, 20\% are reserved for categorization and BDT hyperparameter optimization, and the final 20\% are reserved for validation. 60\% of the NTI events are used for training, 20\% are reserved for hyperparameter optimization, and the remaining 20\% are reserved for testing.

The input variables chosen are: 

\begin{itemize}
\item $p_T/m_{\gamma \gamma}$, $\eta$ and $\phi$ of the two photons. Photon $p_{T}$ is scaled by $m_{\gamma \gamma}$ to reduce unwanted sculpting of the diphoton mass spectrum. 
\item $p_T$, $\eta$, $\phi$ and E of the six jets with highest $p_{T}$
\item Boolean b-tag flag (77\% working point) for each of the six jets with highest $p_{T}$
\item \MET and direction of \MET
\end{itemize}

Distributions of the BDT input variables using $79.8 fb^{-1}$ of data are shown in figures XXX through XXX.

\subsubsection{Leptonic Region} \label{sec:SBBDTlep} 

As in the hadronic region, in the leptonic region, 60\% of the $ttH$ Monte Carlo signal events are used for training, 20\% are reserved for categorization and BDT hyperparameter optimization, and the final 20\% are reserved for validation. 75\% of the NTI events containing zero b-jets but at least one un-tagged jet wiare used for training and the remaining 25\% are reserved for hyperparameter optimization, while 50\% of the NTI events containing one or more b-jets are used for categorization and the remaining 50\% are reserved for testing.

However, due to lower statistics in the leptonic top decay channel due to the smaller top-quark branching ratio to leptons, two cuts are relaxed for the development of the leptonic BDT:

\begin{itemize}
\item The relative $p_{T}$ cuts are loosened from $\frac{p_{T}}{m_{\gamma\gamma} > 0.35$ for the leading photon and $\frac{p_{T}}{m_{\gamma\gamma} > 0.25$ for the subleading photon to a flat $p_{T} > 35$ GeV for the leading photon and $p_{T} > 25$ GeV for the subleading photon.
\item The diphoton invariant mass window is extended from $105 GeV < m_{\gamma \gamma} < 160 GeV$ to $80 GeV < m_{\gamma \gamma} < 250 GeV$.
\end{itemize}

The cuts are again tightened to define the signal region after BDT training is complete- that is, the loosening of these cuts is only utilized to increase BDT training statistics, and does not carry through to other stages of the analysis.

The input variables chosen are: 

\begin{itemize}
\item $p_T/m_{\gamma \gamma}$, $\eta$ and $\phi$ of the two photons. Photon $p_{T}$ is scaled by $m_{\gamma \gamma}$ to reduce unwanted sculpting of the diphoton mass spectrum.
\item $p_T$, $\eta$ and $\phi$ of up to two leptons. 
\item $p_T$, $\eta$, $\phi$ and E of the four jets with highest $p_{T}$
\item Boolean b-tag flag (77\% working point) for each of the four jets with highest $p_{T}$
\item \MET and direction of \MET
\end{itemize}

Distributions of the BDT input variables using $79.8 fb^{-1}$ of data are shown in figures XXX through XXX.

The BDT output distributions are shown in figures XXX through XXX.

We note that the SBBDT performance is independent of $\alpha$; this can be seen in figures XXX through XXX.

\subsection{CP-Sensitive Observables}

In order to train a BDT to discriminate between CP-even and CP-odd $ttH+tH$, we must plot variables at truth-level and investigate their dependence on $\alpha$. We use the HC model $ttH$ and $tH$ samples with alternative values of $\alpha$ generated using MadGraph5_aMCatNLO, added according to their calculated cross-sections given in table XXX.

From these plots, we observe that the strongest variable is the Higgs boson $p_{T}$ : CP-odd $ttH$ and $tH$ have a much more boosted Higgs than CP-even $ttH$ and $tH$, and are more central in $\eta$. Similarly, the angular separation $\Delta \eta$ between the top and anti-top is much larger in CP-odd $ttH$, while the top and anti-top are more back-to-back in azimuthal angle $\Delta \phi$ in CP-even $ttH$ than in CP-odd $ttH$. In $tHjb$ events, we note that the top $p_{T}$ and $\eta$ also have discriminatory power. The mass of the Higgs + leading top system also offers discriminatory power- for $ttH$ and $tWH$ events it increases with $\alpha$, while for $tHjb$ events it decreases with $\alpha$ .

These variables are shown in Figures XXX through XXX.










