
This is a chapter. It includes examples of a section, a subsection, and a subsubsection. You can reference chapters, sections, subsections, and subsubsections with the \texttt{ \textbackslash ref} command: Chapter~\ref{chap:example_chapter}, Section~\ref{sec:example_section}, Subsection~\ref{sec:example_subsection}, Subsubsection~\ref{sec:example_subsubsection}. Your chapters, sections, and subsections will all appear in the table of contents. Subsubsections will not appear. 

This thesis package will generate a list of tables and figures. You can also generate a list of acronyms using the \texttt{ \textbackslash ac} command. To do so, include the acronym definition in the \texttt{ \textbackslash abbreviations} section in the \texttt{thesis.tex} file. The acronym should be called using the \texttt{ \textbackslash ac} command within the body of the document. For example, an abbreviation for the word \ac{TH} has been defined in \texttt{thesis.tex} using the command \texttt{ \textbackslash acro\{TH\}\{Thesis\}}. The first instance of this abbreviated word will be spelled in full, while following instances will be shown as \ac{TH}. By default, Latex will add an ``s'' to the abbreviation in order to make it plural, as in ``Thesiss.'' In order to use a more complicated plural form, the pluralization needs to be defined explicitly in the \texttt{ \textbackslash abbreviations} section as follows: \texttt{ \textbackslash acrodefplural\{TH\}\{Theses\} }. To use the plural form of the abbreviation, use the \texttt{ \textbackslash ac\{TH\} } command in the body. Note that only abbreviations which are used in the body of the document will be included in the table of abbreviations. 


\section{A Section} \label{sec:example_section} 

This is a section. 

\subsection{A Subsection} \label{sec:example_subsection} 

This is a subsection. It will be included in the table of contents. 

\subsubsection{A Subsubsection} \label{sec:example_subsubsection} 

This is a subsubsection. It will not be included in the table of contents. 

